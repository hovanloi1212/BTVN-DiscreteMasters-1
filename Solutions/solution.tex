\documentclass[a4paper]{article}
\usepackage{vntex}
%\usepackage[english,vietnam]{babel}
%\usepackage[utf8]{inputenc}

%\usepackage[utf8]{inputenc}
%\usepackage[francais]{babel}
\usepackage{a4wide,amssymb,epsfig,latexsym,array,hhline,fancyhdr}

\usepackage{amsmath}
\usepackage{amsthm}
\usepackage{multicol,longtable,amscd}
\usepackage{diagbox}%Make diagonal lines in tables
\usepackage{booktabs}
\usepackage{alltt}
\usepackage[framemethod=tikz]{mdframed}% For highlighting paragraph backgrounds
\usepackage{caption,subcaption}

\usepackage{lastpage}
\usepackage[lined,boxed,commentsnumbered]{algorithm2e}
\usepackage{enumerate}
\usepackage{color}
\usepackage{graphicx}							% Standard graphics package
\usepackage{array}
\usepackage{tabularx, caption}
\usepackage{multirow}
\usepackage{multicol}
\usepackage{rotating}
\usepackage{graphics}
\usepackage{geometry}
\usepackage{setspace}
\usepackage{epsfig}
\usepackage{tikz}
\usetikzlibrary{arrows,snakes,backgrounds}
\usepackage[unicode]{hyperref}
\hypersetup{urlcolor=blue,linkcolor=black,citecolor=black,colorlinks=true} 
%\usepackage{pstcol} 								% PSTricks with the standard color package

%\usepackage{fancyhdr}
\setlength{\headheight}{40pt}
\pagestyle{fancy}
\fancyhead{} % clear all header fields
\fancyhead[L]{
 \begin{tabular}{rl}
    \begin{picture}(25,15)(0,0)
    \put(0,-8){\includegraphics[width=8mm, height=8mm]{Images/hcmut.png}}
    %\put(0,-8){\epsfig{width=10mm,figure=hcmut.eps}}
   \end{picture}&
	%\includegraphics[width=8mm, height=8mm]{hcmut.png} & %
	\begin{tabular}{l}
		\textbf{\bf \ttfamily Trường Đại Học Bách Khoa Tp.Hồ Chí Minh}\\
		\textbf{\bf \ttfamily Khoa Khoa Học và Kỹ Thuật Máy Tính}
	\end{tabular} 	
 \end{tabular}
}
\fancyhead[R]{
	\begin{tabular}{l}
		\tiny \bf \\
		\tiny \bf 
	\end{tabular}  }
\fancyfoot{} % clear all footer fields
\fancyfoot[L]{\scriptsize \ttfamily Bài tập về nhà môn Cấu trúc Rời rạc cho KHMT}
\fancyfoot[R]{\scriptsize \ttfamily Trang {\thepage}/\pageref{LastPage}}
\renewcommand{\headrulewidth}{0.3pt}
\renewcommand{\footrulewidth}{0.3pt}


%%%
\setcounter{secnumdepth}{4}
\setcounter{tocdepth}{3}
\makeatletter
\newcounter {subsubsubsection}[subsubsection]
\renewcommand\thesubsubsubsection{\thesubsubsection .\@alph\c@subsubsubsection}
\newcommand\subsubsubsection{\@startsection{subsubsubsection}{4}{\z@}%
                                     {-3.25ex\@plus -1ex \@minus -.2ex}%
                                     {1.5ex \@plus .2ex}%
                                     {\normalfont\normalsize\bfseries}}
\newcommand*\l@subsubsubsection{\@dottedtocline{3}{10.0em}{4.1em}}
\newcommand*{\subsubsubsectionmark}[1]{}
\makeatother

\sloppy
\captionsetup[figure]{labelfont={small,bf},textfont={small,it},belowskip=-1pt,aboveskip=-9pt}
%space remove between caption, figure, and text
\captionsetup[table]{labelfont={small,bf},textfont={small,it},belowskip=-1pt,aboveskip=7pt}
%space remove between caption, table, and text

%\floatplacement{figure}{H}%forced here float placement automatically for figures
%\floatplacement{table}{H}%forced here float placement automatically for table
%the following settings (11 lines) are to remove white space before or after the figures and tables
%\setcounter{topnumber}{2}
%\setcounter{bottomnumber}{2}
%\setcounter{totalnumber}{4}
%\renewcommand{\topfraction}{0.85}
%\renewcommand{\bottomfraction}{0.85}
%\renewcommand{\textfraction}{0.15}
%\renewcommand{\floatpagefraction}{0.8}
%\renewcommand{\textfraction}{0.1}
\setlength{\floatsep}{5pt plus 2pt minus 2pt}
\setlength{\textfloatsep}{5pt plus 2pt minus 2pt}
\setlength{\intextsep}{10pt plus 2pt minus 2pt}

\begin{document}

\begin{titlepage}
\begin{center}
ĐẠI HỌC QUỐC GIA THÀNH PHỐ HỒ CHÍ MINH \\
TRƯỜNG ĐẠI HỌC BÁCH KHOA \\
KHOA KHOA HỌC - KỸ THUẬT MÁY TÍNH 
\end{center}

\vspace{1cm}

\begin{figure}[h!]
\begin{center}
\includegraphics[width=3cm]{Images/hcmut.png}
\end{center}
\end{figure}

\vspace{1cm}


\begin{center}
\begin{tabular}{c}
\multicolumn{1}{l}{\textbf{{\Large CẤU TRÚC RỜI RẠC CHO KHMT}}}\\
~~\\
\hline
\\
\textbf{{\Large Nhóm: Discrete Masters}}\\
\\
\textbf{{\Huge Bài tập về nhà}} \\ \\ \\

\hline
\end{tabular}
\end{center}

\vspace{1.5cm}

\begin{table}[h]
\begin{tabular}{rrl} 
\hspace{5 cm} & SV thực hiện: & Nguyễn Thành Lưu -- 1813017 (Nhóm trưởng) \\
& & Lê Khắc Minh Đăng -- 88471475 \\
& & Bùi Ngô Hoàng Long -- 36811334 \\
& & Lê Bá Thông -- 97501334 \\
& & Hồ Văn Lợi -- 12341334 \\
\end{tabular}
\end{table}
\vspace{1.5cm}
\end{titlepage}

\tableofcontents
\newpage
\section{DS\_propositionallogic.pdf}
\subsection{Bài tập bắt buộc}
\subsubsection{Bài tập 1}
\textbf{Đề bài:} 
\\\ \\\
\textbf{Lời giải:} \\\ \\\
\clearpage
\subsubsection{Bài tập 2}
\textbf{Đề bài:} 
\\\ \\\
\textbf{Lời giải:} \\\ \\\
\clearpage
\subsubsection{Bài tập 3}
\textbf{Đề bài:} 
\\\ \\\
\textbf{Lời giải:} \\\ \\\
\clearpage
\subsubsection{Bài tập 4}
\textbf{Đề bài:} 
\\\ \\\
\textbf{Lời giải:} \\\ \\\
\clearpage
\subsubsection{Bài tập 5}
\textbf{Đề bài:} 
\\\ \\\
\textbf{Lời giải:} \\\ \\\
\clearpage
\subsubsection{Bài tập 6}
\textbf{Đề bài:} 
\\\ \\\
\textbf{Lời giải:} \\\ \\\
\clearpage
\subsubsection{Bài tập 7}
\textbf{Đề bài:} Evaluate each of these expressions
a) $11000 \land (01011 \lor 11011)$
b) $(01111 \land 10101) \lor 01000$
\\\ \\\
\textbf{Lời giải:} \\\ \\\
\begin{enumerate}
	\item [a)]
	$11000 \land (01011 \lor 11011)$\\$\equiv11000 \land 11011$\\$\equiv11000$
	\item [b)]
	$(01111 \land 10101) \lor 01000$\\$\equiv 00101 \lor 01000$\\ $\equiv 01101$\\
\end{enumerate}
\clearpage
\subsubsection{Bài tập 8}
\textbf{Đề bài:} Show that each of these conditional statements is a tautology by using truth tables.
\\\ \\\
\textbf{Lời giải:} \\\ \\\
	\begin{table} [h]
	\flushleft a) $(p \land q) \implies p$\\
	\centering
		\begin{tabular} {|c|c|c|c|}
			\hline
			p&q&$p \land q$&$(p \land q) \implies p$\\
			\hline
			1&1&1&1\\
			1&0&0&1\\
			0&1&0&1\\
			0&0&0&1\\
			\hline
		\end{tabular}
	\\
	\flushleft b) $p \implies (p \lor q)$\\
	\centering
		\begin{tabular}{|c|c|c|c|}
			\hline
			p&q&$p \lor q$&$p \implies (p \lor q)$\\
			\hline
			1&1&1&1\\
			1&0&1&1\\
			0&1&1&1\\
			0&0&0&1\\
			\hline
		\end{tabular}
	\\
	\flushleft	c)  $\lnot p \implies (p \implies q) $\\ 
	\centering
		\begin{tabular} {|c|c|c|c|c|}
			\hline
			p&q&$\lnot p$&$p \implies q$&$\lnot p \implies (p \implies q)$\\
			\hline
			1&1&0&1&1\\
			1&0&0&0&1\\
			0&1&1&1&1\\
			0&0&1&1&1\\	
			\hline
		\end{tabular}
	\\
	\flushleft	d)   $(p \land q) \implies (p \implies q)$\\
	\centering
		\begin{tabular} {|c|c|c|c|c|}
			\hline
			p&q&$p \land q$&$p \implies q$&$(p \land q) \implies (p \implies q)$\\
			\hline
			1&1&1&1&1\\
			1&0&0&0&1\\
			0&1&0&1&1\\
			0&0&0&1&1\\
			\hline
		\end{tabular}
	\\
	\flushleft	e)   $\lnot (p \implies q) \implies p $\\
	\centering
		\begin{tabular} {|c|c|c|c|c|}
			\hline
			p&q&$p \implies q$&$\lnot (p \implies q)$&$\lnot (p \implies q) \implies p$\\
			\hline
			1&1&1&0&1\\
			1&0&0&1&1\\
			0&1&1&0&1\\
			0&0&1&0&1\\
			\hline
		\end{tabular}
	\\
	\flushleft	f)   $\lnot(p \implies q) \implies \lnot q $\\
	\centering
		\begin{tabular} {|c|c|c|c|c|c|}
			\hline
			p&q&$\lnot q$&$p \implies q$&$\lnot (p \implies q)$&$\lnot (p \implies q) \implies \lnot q$\\
			\hline
			1&1&0&1&0&1\\
			1&0&1&0&1&1\\
			0&1&0&1&0&1\\
			0&0&1&1&0&1\\
			\hline
		\end{tabular}
	\\
	\flushleft	g)   $[\lnot p \land (p \lor q )] \implies q $\\
	\centering
		\begin{tabular} {|c|c|c|c|c|c|}
			\hline
			p&q&$\lnot p$&$p \lor q$&$\lnot p \land (p \lor q)$&$[\lnot p \land (p \lor q )] \implies q $\\
			\hline
			1&1&0&1&0&1\\
			1&0&0&1&0&1\\
			0&1&1&1&1&1\\
			0&0&1&0&0&1\\
			\hline
		\end{tabular}
	\\	
	\flushleft	h)  $[(p \implies q) \land (q \implies r)] \implies (p \implies r)$\\
	\centering
		\begin{tabular} {|c|c|c|c|c|c|c|c|}
			\hline
			p&q&r&$p \implies q$&$q \implies r$&$(p \implies q) \land (q \implies r)$&$p \implies r$&$[(p \implies q) \land (q \implies r)] \implies (p \implies r)$\\
			\hline
			1&1&1&1&1&1&1&1\\
			1&1&0&1&0&0&0&1\\
			1&0&1&0&1&0&1&1\\
			1&0&0&0&1&0&0&1\\
			0&1&1&1&1&1&1&1\\
			0&1&0&1&0&0&1&1\\
			0&0&1&1&1&0&1&1\\
			0&0&0&1&1&0&1&1\\
			\hline
		\end{tabular}
	\end{table}
\clearpage
\subsubsection{Bài tập 9}
\textbf{Đề bài:} Show that these compound propositionals are logically equivalent.
\\\ \\\
\textbf{Lời giải:} \\\ \\\
	\begin{enumerate}
	\item[a)] $\lnot (p \iff q)$ và $\lnot p \iff q$
	\begin{table} [h]
		\centering
		\begin{tabular} {|c|c|c|c|c|c|c|}
			\hline
			$p$&$q$&$\lnot p$&$  p \iff q$&$\lnot (p \iff q)$&$ \lnot p \iff q$& $\lnot (p \iff q) \vDash \lnot p \iff q$\\
			\hline
			1&0&0&0&0&1&1\\
			1&1&0&1&0&0&1\\
			0&1&1&0&1&1&1\\
			0&0&1&1&0&0&1\\
			\hline
		\end{tabular}
	\end{table}
	\item[b)]	$(p \implies q) \land (p \implies r) $ và $ p \implies (q \land r)$\\
				$(p \implies q) \land (p \implies r)$\\
				$\equiv (\lnot p \lor q) \land (\lnot p \lor r)$\\
				$\equiv  \lnot p \land (q \lor r)$\\
				$\equiv  p \implies (q \lor r)$\\
	\item[c)]   $(p\implies r) \land (q \implies r) $ và $ (p \lor q) \implies r$\\ 
				$(p\implies r) \land (q \implies r)$\\
				$\equiv (\lnot p \lor r) \land (\lnot q \lor r)$\\
				$\equiv (\lnot q \land \lnot p) \lor r$\\
				$\equiv  \lnot(\lnot p \land \lnot q) \implies r$\\
				$\equiv (p \lor q) \implies r$\\
	\item[d)]	$(p \implies q) \lor (p \implies r) $ và $ p \implies (q \lor r)$\\
				$(p \implies q) \lor (p \implies r)$\\
				$\equiv  \lnot p \lor q \lor \lnot p \lor r$\\
				$\equiv  \lnot p \lor (q \lor r)$\\ $\equiv  p \implies (q \lor r)$\\
	\item[e)]	$\lnot p \implies (q \implies r) $ và $ q \implies (p \lor r)$\\
				$\lnot p \implies (q \implies r)$\\
				$\equiv  \lnot p \implies (\lnot q \lor r)$\\
				$\equiv  \lnot (\lnot p) \lor \lnot q \lor r)$\\
				$\equiv p \lor \lnot q \lor r$\\
				$\equiv  \lnot q \lor p \lor r $\\
				$\equiv  q \implies (p \lor r)$\\
	\item[f)] $p \iff q $ và $ (p \implies q) \land (q \implies p)$
	\begin{table} [h]
		\centering
		\begin{tabular} {|c|c|c|c|c|c|c|}
			\hline
			$p$&$q$&$p \implies q$&$q \implies p$&$(p \implies q) \land (q \implies p)$ &$p \iff q$ &$p \iff q \vDash (p \implies q) \land (q \implies p)$\\
			\hline
			1&1&1&1&1&1&1\\
			1&0&0&1&0&0&1\\
			0&1&1&0&0&0&1\\
			0&0&1&1&1&1&1\\
			\hline
		\end{tabular}
	\end{table}
\end{enumerate}
\clearpage
\subsubsection{Bài tập 10}
\textbf{Đề bài: } Show that these compound propositionals are logically equivalent by developing a series of logical equivalences \\\ \\\
a) $\lnot(p\rightarrow (\lnot q \land r))$ and $p \land (q \lor \lnot r)$.\\\
b) $\lnot[(p \land (q\lor r)) \land (\lnot p \lor \lnot q \lor r)]$ and $\lnot p \lor \lnot r$. \\\
c) $\lnot [[[[(p \land q)\land r] \lor [(p \land r) \land \lnot r]] \lor \lnot q] \rightarrow s]$ and $[(p \land r) \lor \lnot q] \land \lnot s$. \\\ \\\
\textbf{Lời giải:} \\\ \\\
a) Ta có: \\\
$\lnot(p\rightarrow (\lnot q \land r)) \\\equiv \lnot(\lnot p \lor(\lnot q \land r)) \\\equiv p \land \lnot (\lnot q \land r) \\\equiv p \land (q \lor \lnot r)$ \\\ \\\
b) Ta có: \\\
$\lnot[(p \land (q\lor r)) \land (\lnot p \lor \lnot q \lor r)] \\\equiv \lnot (p \land (q\lor r)) \lor \lnot(\lnot p \lor \lnot q \lor r) \\\equiv (\lnot p \lor \lnot (q\lor r)) \lor (p \land q \land \lnot r) \\\equiv \lnot p \lor (p \land q \land \lnot r) \lor (\lnot q \land \lnot r) \\\equiv ((\lnot p \lor p) \land (\lnot p \lor (q \land \lnot r))) \lor (\lnot q \land \lnot r) \\\equiv (\textbf{T} \land (\lnot p \lor (q \land \lnot r))) \lor (\lnot q \land \lnot r) \\\equiv \lnot p \lor (q \land \lnot r) \lor (\lnot q \land \lnot r) \\\equiv \lnot p \lor (\lnot r \land (q \lor \lnot q)) \\\equiv \lnot p \lor (\lnot r \land \textbf{T}) \\\equiv \lnot p \lor \lnot r$ \\\ \\\
c) Ta có: \\\
$\lnot [[[[(p \land q)\land r] \lor [(p \land r) \land \lnot r]] \lor \lnot q] \rightarrow s] \\\equiv \lnot [[(p \land q \land r) \lor (p \land (r \land \lnot r)) \lor \lnot q] \rightarrow s] \\\equiv \lnot [[(p \land q \land r) \lor (p \land \textbf{F}) \lor \lnot q] \rightarrow s] \\\equiv \lnot [[(p \land q \land r) \lor \textbf{F} \lor \lnot q] \rightarrow s]\\\equiv \lnot [[(p \land q \land r) \lor \lnot q] \rightarrow s]\\\equiv \lnot [\lnot [(p \land q \land r) \lor \lnot q] \lor s] \\\equiv [(p \land q \land r) \lor  \lnot q]\land \lnot s\\\equiv [[(p \land r) \lor  \lnot q] \land (q \lor \lnot q)]\land \lnot s\\\equiv [[(p \land r) \lor  \lnot q] \land \textbf{T}]\land \lnot s \\\equiv [(p \land r) \lor \lnot q] \land \lnot s$

\clearpage
\subsubsection{Bài tập 11}
\textbf{Đề bài:} You cannot edit a protected Wikipedia entry unless you are an administrator. Express your answer in terms of $e$: “You can edit a protected Wikipedia entry” and $a$: “You are an administrator.” \\\ \\\
\textbf{Lời giải:} \\\ \\\
Ta có thể biểu diễn sang: $\lnot a \rightarrow \lnot e$.
\clearpage
\subsubsection{Bài tập 12}
\textbf{Đề bài:} You can see the movie only if you are over 18 years old or you have the permission of a parent. Express your answer in terms of $m$: “You can see the movie,” $e$: “You are over 18 years old,” and $p$: “You have the permission of a parent.” \\\ \\\
\textbf{Lời giải:} \\\ \\\
Ta có thể biểu diễn sang: $\lnot (e \lor p) \rightarrow \lnot m$.
\clearpage
\subsubsection{Bài tập 13}
\textbf{Đề bài:} You can graduate only if you have completed the requirements of your major and you do not owe money to the university and you do not have an overdue library book. Express your answer in terms of
$g$: “You can graduate,” $m$: “You owe money to the university,” $r$: “You have completed the requirements
of your major,” and $b$: “You have an overdue library book.” \\\ \\\
\textbf{Lời giải:} \\\ \\\
Ta có thể biểu diễn sang: $(m \lor \lnot r \lor b) \rightarrow \lnot g$.
\clearpage
\subsubsection{Bài tập 14}
\textbf{Đề bài:} 
\\\ \\\
\textbf{Lời giải:} \\\ \\\
\clearpage
\subsubsection{Bài tập 15}
\textbf{Đề bài:} 
\\\ \\\
\textbf{Lời giải:} \\\ \\\
\clearpage
\subsubsection{Bài tập 16}
\textbf{Đề bài:} 
\\\ \\\
\textbf{Lời giải:} \\\ \\\
\clearpage
\subsubsection{Bài tập 17}

\clearpage
\clearpage

\section{New\_Homework01\_Propositional\_Logic.pdf}
\subsection{Bài tập bắt buộc}
\subsubsection{Bài tập 1}
\textbf{Đề bài:} 
\\\ \\\
\textbf{Lời giải:} \\\ \\\
\clearpage
\subsubsection{Bài tập 2}
\textbf{Đề bài:} 
\\\ \\\
\textbf{Lời giải:} \\\ \\\
\clearpage
\subsubsection{Bài tập 3}
\textbf{Đề bài:} 
\\\ \\\
\textbf{Lời giải:} \\\ \\\
\clearpage
\subsubsection{Bài tập 4}
\textbf{Đề bài:} 
\\\ \\\
\textbf{Lời giải:} \\\ \\\
\clearpage
\subsubsection{Bài tập 5}
\textbf{Đề bài:} show that $[p \land (p \implies q)] \implies q$ is a tautology using truth table.
\\\ \\\
\textbf{Lời giải:} \\\ \\\
$[p \land (p \implies q)] \implies q$\\
\begin{table} [h]
	\centering
	\begin{tabular} {|c|c|c|c|c|}
		\hline
		$p$&$q$&$p \implies q$& $p \land (p \implies q)$ &$[p \land (p \implies q)]\implies q$\\
		\hline
		1&1&1&1&1\\
		\hline
		1&0&0&0&1\\
		\hline
		0&1&1&0&1\\
		\hline
		0&0&1&0&1\\
		\hline
	\end{tabular}
\end{table}
\clearpage
\subsubsection{Bài tập 6}
\textbf{Đề bài:} Prove that $(p \implies q) \lor (p \implies r)$ và $p \implies (q \lor r)$ are logically equipvalent( without using this equipvalent from the table).
\\\ \\\
\textbf{Lời giải:} \\\ \\\
$(p \implies q) \lor (p \implies r)$ và $p \implies (q \lor r)$
$(p \implies q) \lor (p \implies r)$\\
$\equiv (\lnot p \lor q) \lor (\lnot p \lor r)$\\
$\equiv \lnot p \lor (q \lor r)$\\
$\equiv p \implies (q \lor r)$\\
\clearpage
\subsubsection{Bài tập 7}
\textbf{Đề bài: }Find an assignment of the variables $p, q, r$ such that the proposition $(p \lor \lnot q) \land (p \lor q) \land (q \lor r) \land (q \lor \lnot r) \land (r \lor \lnot p) \land (r \lor p)$ is satisfied. For a bonus 5 points, prove that this assignment is unique. \\\ \\\
\textbf{Lời giải:} \\\ \\\
Khi $p$ đúng, $q$ đúng và $r$ đúng thì mệnh đề trên thoả mãn. \\\ \\\
* Chứng minh bộ ba $p,q,r$ làm cho mệnh đề đúng là duy nhất: \\\
Ta có: \\\
$(p \lor \lnot q) \land (p \lor q) \land (q \lor r) \land (q \lor \lnot r) \land (r \lor \lnot p) \land (r \lor p) \\\equiv  (p \lor (\lnot q \land q)) \land (q \lor (\lnot r \land r)) \land (r \lor (\lnot p \land p))\\\equiv (p \lor \textbf{F}) \land (q \lor \textbf{F}) \land (r \lor \textbf{F}) \\\equiv p \land q \land r$ \\\
Mệnh đề này đúng khi và chỉ khi cả ba biến $p,q,r$ đều nhận chân trị đúng. \\\
Ta có điều phải chứng minh.
\clearpage
\subsubsection{Bài tập 8}
\textbf{Đề bài:} 
\\\ \\\
\textbf{Lời giải:} \\\ \\\
\clearpage
\subsection{Bonus}
\textbf{Bài tập 1.2 19-23}
\begin{enumerate}
	\item[19)] A is a Knight, B is a Knave.
	\item[20)] A is a Knave, B is a Knight.
	\item[21)] A,B are Knights.
	\item[22)] A,B are independent to each other  so they can be a Knight or Knave. We can not conclude anything in more details.
	\item[23)] A is a Knave, B is a Knight.
\end{enumerate}
\textbf{Bài tập 1.2 24-31}
\begin{enumerate}
	\item[24)] We have two case:\\+Case(1):A is a Knight, B is a Spy, C is a Knave.\\+Case(2): A is a Spy, B is a Knight, C is a Knave.
	\item[25)] A is a Knight, B is a Spy, C is a Knave
	\item[26)] In this situation, there can't be a Knave or Knight in one of them.So we have no solution.
	\item[27)] A is a Knight, B is a Spy, C is a Knave.
	\item[28)] A is a Knave, B is a Spy, C is a Knight.
	\item[29)] Any of them can be a Knight, Knave or Spy.
	\item[30)] We have two case:\\+Case(1):A is a Knight, B is a Spy, C is a Knave.\\+Case(2): A is a Spy, B is a Knight, C is a Knave.
	\item[31)] We can't determine any solution.
\end{enumerate}
\textbf{Bài tập 1.2.40} Find the output of each of these combinatorial circuits.\\
\begin{enumerate}
	\item[a)] $\lnot p \lor  \lnot q$
	\item[b)] $(p \lor (\lnot p \land q))$	
\end{enumerate}	
\textbf{Bài tập 1.2.41} Find the output of each of these combinatorial circuits.\\
\begin{enumerate}
	\item[a)] $\lnot(p \land (q \lor \lnot r))$
	\item[b)] $(\lnot p \land \lnot q) \lor (p \land r)$	
\end{enumerate}
\textbf{Bài tập 1.2.42} Construct a combinatorial circuit using inverters, OR gates, and AND gates that produces the output $(p \lor \lnot r) \lor (\lnot q lor r)$ from input bits p, q, and r.
\textbf{Bài tập 1.2.43} Construct a combinatorial circuit using inverters, OR gates, and AND gates that produces the output $((\lnot p \lor \lnot r) \lor \lnot p) \lor (\lnot p \land (p \lor r))$\\ from input bits p, q, and r.
\textbf{Bài tập 1.3.8:}  Use De Morgan’s laws to find the negation of each of the following statements.
\begin{enumerate}
	\item[a)] Kwame will take a job in industry or go to graduate school.
				A(x): x will take a job in industry.\\
				B(x): x will go to graduate school.\\
				Domains for x all human being.\\
				A(Kwame) $\lor$ B(Kwame)\\
				$\rightarrow \lnot (A(Kwame) \lor B(Kwame))$\\
				$\equiv \lnot A(Kwame) \land \lnot B(Kwame)$\\
				Kwame won't take a job in industry amd won't go to graduate school.
	\item[b)] Yoshiko knows Java and calculus.
				A(x): x knows Java.\\
				B(x): x knows calculus.\\
				A(Yoshiko) $\land$ B(Yoshiko).\\
				$\rightarrow \lnot (A(Yoshiko) \land B(Yoshiko))$\\
				$\equiv \lnot A(Yoshiko) \lor \lnot B(Yoshiko).$\\
				Yoshiko doesn't know Java or Calculus.\\
	\item[c)] James is young and strong.
				A(x): x is young.\\
				B(x): x is strong.\\
				A(James) $\land$ B(James).\\
				$\rightarrow \lnot (A(James) \land B(James)$\\
				$\equiv \lnot A(James) \lor \lnot B(James).$\\
				James isn't strong or young.\\
	\item[d)] Rita will move to Oregon or Washington.
				A(x): x is move to Oregon.\\
				B(x): x is move to Washington.\\
				A(Rita) $\lor$ B(Rita).\\
				$\rightarrow \lnot (A(x) \lor B(x))$\\
				$\equiv \lnot A(Rita) \land \lnot B(Rita)$.\\
				Rita doesn't go to Oregon or Washington.\\
\end{enumerate}
\textbf{Bài tập 1,3.9:} Show that each of these conditional statements is a tautology by using truth tables.
\begin{enumerate}
	\begin{table} [h]
		\item[a)] $(p \land q) \implies p$
		\centering
		\begin{tabular} {|c|c|c|c|}
			\hline
			p&q&$p \land q$&$(p \land q) \implies p$\\
			\hline
			1&1&1&1\\
			1&0&0&1\\
			0&1&0&1\\
			0&0&0&1\\
			\hline
		\end{tabular}
		\item[b)] $p \implies (p \lor q)$
		\centering
		\begin{tabular} {|c|c|c|c|}
			\hline
			p&q&$p \lor q$&$p \implies q(p \lor q)$\\
			\hline
			1&1&1&1\\
			1&0&1&1\\
			0&1&1&1\\
			0&0&0&1\\
			\hline
		\end{tabular}
		\item[c)] $\lnot p \implies (p \implies q)$
		\centering
		\begin{tabular} {|c|c|c|c|c|}
			\hline
			p&q&$\lnot p$&$p \implies q$&$\lnot p \implies (p \implies q)$\\
			\hline
			1&1&0&1&1\\
			1&0&0&0&1\\
			0&1&1&1&1\\
			0&0&1&1&1\\
			\hline
		\end{tabular}
		\item[d)] $(p \lor q) \implies p \implies q$
		\centering
		\begin{tabular} {|c|c|c|c|c|}
			\hline
			p&q&$p \lor q$&$p \implies q$&$(p \lor q) \implies p \implies q$\\
			\hline
			1&1&1&1&1\\
			1&0&0&0&1\\
			0&1&0&1&1\\
			0&0&0&1&1\\
			\hline
		\end{tabular}
		\item[e)] $\lnot (p \implies q) \implies p$
		\centering
		\begin{tabular} {|c|c|c|c|c|}
			\hline
			p&q&$p \implies q$&$\lnot (p \implies q)$&$\lnot (p \implies q) \implies p$\\
			\hline
			1&1&1&0&1\\
			1&0&0&1&1\\
			0&1&1&0&1\\
			0&0&1&0&1\\
			\hline
		\end{tabular}
		\item[f)] $\lnot(p \implies q) \implies \lnot q$
		\centering
		\begin{tabular} {|c|c|c|c|c|c|}
			\hline
			p&q&$\lnot q$&$p \implies q$&$\lnot (p \implies q)$&$\lnot(p \implies q) \implies \lnot q$\\
			\hline
			1&1&0&1&0&1\\
			1&0&1&0&1&1\\
			0&1&0&1&0&1\\
			0&0&1&1&0&1\\
			\hline
		\end{tabular}
	\end{table}
\end{enumerate}
\textbf{Bài tập 1.3.10:} Show that each of these conditional statements is a tautology by using truth tables.
\begin{enumerate}
	\begin{table} [h]
		\item[a)] $(\lnot p \land (p \lor q)) \implies q$
		\centering
		\begin{tabular} {|c|c|c|c|c|c|}
			\hline
			p&q&$\lnot p$&$p \lor q$&$\lnot p \land (p \lor q)$&$(\lnot p \land (p \lor q)) \implies q$\\	
			\hline
			1&1&0&1&0&1\\
			1&0&0&1&0&1\\
			0&1&1&1&1&1\\
			0&0&1&0&0&1\\
			\hline
		\end{tabular}
		\item[b)] $[(p \implies q) \land (q \implies r)] \implies (p \implies r)$
		\centering
		\begin{tabular} {|c|c|c|c|c|c|c|c|}
			\hline
			p&q&r&$p \implies q$&$q \implies r$&$(p \implies q) \land (q \implies r)$&$p \implies r$&$[(p \implies q) \land (q \implies r)] \implies (p \implies r)$\\
			\hline
			1&1&1&1&1&1&1&1\\
			1&1&0&1&0&0&0&1\\
			1&0&1&0&1&0&1&1\\
			1&0&0&0&1&0&0&1\\
			0&1&1&1&1&1&1&1\\
			0&1&0&1&0&0&1&1\\
			0&0&1&1&1&1&1&1\\
			0&0&0&1&1&1&1&1\\
			\hline
		\end{tabular}
		\item[c)] $[p \land (p \implies q)] \implies q$
		\centering
		\begin{tabular} {|c|c|c|c|c|}
			\hline
			p&q&$p \implies q$&$p \land (p \implies q)$&$[p \land (p \implies q)] \implies q$\\
			\hline
			1&1&1&1&1\\
			1&0&0&0&1\\
			0&1&1&0&1\\
			0&0&1&0&1\\
			\hline
		\end{tabular}
		\item[d)] $[(p \lor q) \land (p \implies r) \land (q \implies r)] \implies r$
		\centering
		\begin{tabular} {|c|c|c|c|c|c|c|c|c|}
			\hline
			p&q&r&$p \lor q$&$p \implies r$&$q \implies r$&$(p \lor q) \land (p \implies r)$&$(p \lor q) \land (p \implies r) \land (q \implies r)$&$[(p \lor q) \land (p \implies r) \land (q \implies r)] \implies r$\\
			\hline
			1&1&1&1&1&1&1&1&1\\
			1&1&0&1&0&0&0&0&1\\
			1&0&1&1&1&1&1&1&1\\
			1&0&0&1&0&1&0&0&1\\
			0&1&1&1&1&1&1&1&1\\
			0&1&0&1&1&0&1&0&1\\
			0&0&1&0&1&1&0&0&1\\
			0&0&0&0&1&1&0&0&1\\
			\hline			
		\end{tabular}
	\end{table}
\end{enumerate}
\textbf{Bài tập 1.3.12:} Show that each conditional statement here is a tautology without using truth tables: 
\begin{enumerate}[a)]
	\item $[\lnot p \land (p \lor q)] \rightarrow q$
	\item $[(p \rightarrow q) \land (q \rightarrow r)] \rightarrow (p \rightarrow r)$
	\item $[p \land (p \rightarrow q)] \rightarrow q$
	\item $[(p \lor q) \land (p \rightarrow r) \land (q \rightarrow r)] \rightarrow r$
\end{enumerate}
\textbf{Lời giải: }
\begin{enumerate}[a)]
	\item $[\lnot p \land (p \lor q)] \rightarrow q \\\equiv \lnot[\lnot p \land (p \lor q)] \lor q \\\equiv p \lor (\lnot p \land \lnot q) \lor q \\\equiv ((p \lor \lnot p) \land (p \lor \lnot q)) \lor q \\\equiv (\textbf{T} \land (p \lor \lnot q))\lor q \\\equiv p \lor \lnot q \lor q \equiv p \lor \textbf{T} \equiv \textbf{T}$
	\item $[(p \rightarrow q) \land (q \rightarrow r)] \rightarrow (p \rightarrow r) \\\equiv \lnot [(\lnot p \lor q) \land (\lnot q \lor r)] \lor (\lnot p \lor r) \\\equiv (p \land \lnot q) \lor (q \land \lnot r) \lor (\lnot p \lor r) \\\equiv (p \land \lnot q)\lor \lnot p \lor (q \land \lnot r) \lor r \\\equiv \textbf{T} \lor \textbf{T} \equiv \textbf{T}$
	\item $[p \land (p \rightarrow q)] \rightarrow q \\\equiv \lnot(p \land (\lnot p \lor q))\lor q \\\equiv \lnot p \lor (p \land \lnot q) \lor q \\\equiv \textbf{T} \lor q \equiv \textbf{T}$
	\item $[(p \lor q) \land (p \rightarrow r) \land (q \rightarrow r)] \rightarrow r \\\equiv \lnot [(p \lor q) \land (\lnot p \lor r) \land (\lnot q \lor r)] \lor r \\\equiv (\lnot p \land \lnot q) \lor (p \land \lnot r) \lor (q \land \lnot r) \lor r \\\equiv (\lnot p \land \lnot q) \lor (p \land \lnot r) \lor \textbf{T} \equiv \textbf{T}$
\end{enumerate}
\textbf{Bài tập 1.3.16:} Show that $p \leftrightarrow q$ and $(p \land q) \lor (\lnot p \land \lnot q)$ are logically
equivalent. \\\ \\\
\textbf{Lời giải:} \\\ \\\
Ta có : \\\ 
$p \leftrightarrow q \\\equiv (p \rightarrow q) \land (q \rightarrow p) \\\equiv (\lnot p \lor q) \land (\lnot q \lor p) \\\equiv ((\lnot p \lor q)\land \lnot q) \lor ((\lnot p \lor q)\land p) \\\equiv ((\lnot p \land \lnot q) \lor (q \land \lnot q)) \lor ((\lnot p \land p) \lor (q \land p)) \\\equiv (p \land q) \lor (\lnot p \land \lnot q)$ \\\ \\\
\textbf{Bài tập 1.3.17:} Show that $\lnot(p \leftrightarrow q)$ and $p \leftrightarrow \lnot q$ are logically equivalent. \\\ \\\
\textbf{Lời giải:} \\\ \\\
Kết hợp kết quả từ câu 16, Ta có: \\\
$\lnot(p \leftrightarrow q) \\\equiv \lnot((\lnot p \land \lnot q) \lor (p \land q)) \\\equiv (p \lor q) \land (\lnot p \lor \lnot q) \\\equiv (\lnot q \rightarrow p) \land (p \rightarrow \lnot q) \\\equiv p \leftrightarrow \lnot q$ \\\ \\\
\textbf{Bài tập 1.3.18:} Show that $p \rightarrow q$ and $\lnot q \rightarrow \lnot p$ are logically equivalent. \\\ \\\
\textbf{Lời giải: }\\\ \\\
Bảng chân trị: \\\ \\\
\begin{tabular}{|c|c|c|c|c|c|}
\hline 
p & q & $\lnot p$ & $\lnot q$ & $p \rightarrow q$ & $\lnot q \rightarrow \lnot p$ \\ 
\hline 
0 & 0 & 1 & 1 & 1 & 1 \\ 
\hline 
0 & 1 & 1 & 0 & 1 & 1 \\ 
\hline 
1 & 0 & 0 & 1 & 0 & 0 \\ 
\hline 
1 & 1 & 0 & 0 & 1 & 1 \\ 
\hline 
\end{tabular} \\\ \\\
Theo bảng chân trị, ta có điều phải chứng minh. \\\ \\\
\textbf{Bài tập 1.3.19: }Show that $\lnot p \leftrightarrow q$ and $p \leftrightarrow \lnot q$ are logically equivalent. \\\ \\\
\textbf{Lời giải:} \\\ \\\
Bảng chân trị: \\\ \\\
\begin{tabular}{|c|c|c|c|c|c|}
\hline 
p & q & $\lnot p$ & $\lnot q$ & $\lnot p \leftrightarrow q$ & $p \leftrightarrow \lnot q$ \\ 
\hline 
0 & 0 & 1 & 1 & 0 & 0 \\ 
\hline 
0 & 1 & 1 & 0 & 1 & 1 \\ 
\hline 
1 & 0 & 0 & 1 & 1 & 1 \\ 
\hline 
1 & 1 & 0 & 0 & 0 & 0 \\ 
\hline 
\end{tabular} \\\ \\\
Theo bảng chân trị, ta có điều phải chứng minh. \\\ \\\
\textbf{Bài tập 1.3.20: }Show that $\lnot(p \oplus q)$ and $p \leftrightarrow q$ are logically equivalent. \\\ \\\
Bảng chân trị: \\\ \\\
\begin{tabular}{|c|c|c|c|c|}
\hline 
p & q & $p \oplus q$ & $\lnot(p \oplus q)$ & $p \leftrightarrow q$ \\ 
\hline 
0 & 0 & 0 & 1 & 1 \\ 
\hline 
0 & 1 & 1 & 0 & 0 \\ 
\hline 
1 & 0 & 1 & 0 & 0 \\ 
\hline 
1 & 1 & 0 & 1 & 1 \\ 
\hline 
\end{tabular} \\\ \\\
Theo bảng chân trị, ta có điều phải chứng minh. \\\ \\\
\textbf{Bài tập 1.3.21: }Show that $\lnot(p \leftrightarrow q)$ and $\lnot p \leftrightarrow q$ are logically equivalent. \\\ \\\
\textbf{Lời giải: } \\\ \\\
Kết hợp kết quả từ bài 17 và bài 19, ta có: \\\ \\\
$\lnot (p \leftrightarrow q) \\\equiv p \leftrightarrow \lnot q \\\equiv \lnot p \leftrightarrow q$ \\\ \\\
\textbf{Bài tập 1.3.22: }Show that $(p \rightarrow q) \land (p \rightarrow r)$ and $p \rightarrow (q \land r)$ are logically equivalent. \\\ \\\
\textbf{Lời giải: } \\\ \\\
Ta có: \\\
$(p \rightarrow q) \land (p \rightarrow r) \\\equiv (\lnot p \lor q) \land (\lnot p \lor r) \\\equiv \lnot p \lor (q \land r) \\\equiv p \rightarrow (q \land r).$ \\\ \\\
\textbf{Bài tập 1.3.23: }Show that $(p \rightarrow r) \land (q \rightarrow r)$ and $(p \lor q) \rightarrow r$ are logically equivalent. \\\ \\\
\textbf{Lời giải:} \\\ \\\
Ta có: \\\
$(p \rightarrow r) \land (q \rightarrow r) \\\equiv (\lnot p \lor r) \land (\lnot q \lor r) \\\equiv r \lor (\lnot p \land \lnot q) \\\equiv r \lor \lnot (p \lor q)\\\equiv (p \lor q) \rightarrow r.$ \\\ \\\
\textbf{Bài tập 1.3.24: }Show that $(p \rightarrow q) \lor (p \rightarrow r)$ and $p \rightarrow (q \lor r)$ are logically equivalent. \\\ \\\
\textbf{Lời giải:} \\\ \\\
Ta có: \\\
$(p \rightarrow q) \lor (p \rightarrow r) \\\equiv (\lnot p \lor q) \lor (\lnot p \lor r) \\\equiv \lnot p \lor (q \lor r) \\\equiv p \rightarrow (q \lor r).$ \\\ \\\
\textbf{Bài tập 1.3.25: }Show that $(p \rightarrow r) \lor (q \rightarrow r)$ and $(p \land q) \rightarrow r$ are logically equivalent. \\\ \\\
\textbf{Lời giải: } \\\ \\\
Ta có: \\\
$(p \rightarrow r) \lor (q \rightarrow r) \\\equiv (\lnot p \lor r) \lor (\lnot q \lor r) \\\equiv (\lnot p \lor \lnot q) \lor r \\\equiv \lnot (p \land q) \lor r \\\equiv (p \land q) \rightarrow r.$ \\\ \\\
\textbf{Bài tập 1.3.26: }Show that $\lnot p \rightarrow (q \rightarrow r)$ and $q \rightarrow (p \lor r)$ are logically equivalent. \\\ \\\
\textbf{Lời giải: } \\\ \\\
Ta có: \\\
$\lnot p \rightarrow (q \rightarrow r) \\\equiv \lnot (\lnot p) \lor (\lnot q \lor r) \\\equiv \lnot q \lor (p \lor r) \\\equiv q \rightarrow (p \lor r)$ \\\ \\\
\textbf{Bài tập 1.3.27: }Show that $p \leftrightarrow q$ and $(p \rightarrow q) \land (q \rightarrow p)$ are logically equivalent. \\\ \\\
\textbf{Lời giải: }\\\ \\\
Bảng chân trị: \\\ \\\
\begin{tabular}{|c|c|c|c|c|c|}
\hline 
p & q & $p \rightarrow q$ & $q \rightarrow p$ & $(p \rightarrow q) \land (q \rightarrow p)$ & $p \leftrightarrow q$ \\ 
\hline 
0 & 0 & 1 & 1 & 1 & 1 \\ 
\hline 
0 & 1 & 1 & 0 & 0 & 0 \\ 
\hline 
1 & 0 & 0 & 1 & 0 & 0 \\ 
\hline 
1 & 1 & 1 & 1 & 1 & 1 \\ 
\hline 
\end{tabular} \\\ \\\
Theo bảng chân trị, ta có điều phải chứng minh. \\\ \\\
\textbf{Bài tập 1.3.28: }Show that $p \leftrightarrow q$ and $\lnot p \leftrightarrow \lnot q$ are logically equivalent. \\\ \\\
\textbf{Lời giải: } \\\ \\\
Ta có: \\\ 
$p \leftrightarrow q \\\equiv (p \rightarrow q) \land (q \rightarrow p) \\\equiv (\lnot q \rightarrow \lnot p) \land (\lnot p \rightarrow \lnot q) \\\equiv \lnot p \leftrightarrow \lnot q.$ \\\ \\\
\textbf{Bài tập 1.3.32: }Show that $(p \land q) \rightarrow r$ and $(p \rightarrow r) \land (q \rightarrow r)$ are not logically equivalent. \\\ \\\
\textbf{Lời giải: } \\\ \\\ 
Với p, r nhận chân trị False và q nhận chân trị True, ta có: \\\ 
\begin{enumerate}
	\item $(p \land q) \rightarrow r \equiv (\textbf{F} \land \textbf{T}) \rightarrow \textbf{F} \equiv \textbf{F} \rightarrow \textbf{F} \equiv \textbf{T}.$ 
	\item $(p \rightarrow r) \land (q \rightarrow r) \equiv (\textbf{F} \rightarrow \textbf{F}) \land (\textbf{T} \rightarrow \textbf{F}) \equiv \textbf{T} \land \textbf{F} \equiv \textbf{F}.$
\end{enumerate}
Hai giá trị này không tương đương logic với nhau, do đó $(p \land q) \rightarrow r$ và $(p \rightarrow r) \land (q \rightarrow r)$ không tương đương logic với nhau (đpcm). \\\ \\\
\textbf{Bài tập 1.3.39: }Why are the duals of two equivalent compound propositions also equivalent, where these compound propositions contain only the operators $\land$, $\lor$, and $\lnot$? \\\ \\\
\textbf{Lời giải: } \\\ \\\
Gọi p và q là hai mệnh đề ghép tương đương, trong đó p và q chỉ chứa các toán tử $\land$, $\lor$, và $\lnot$. Vì $p$ và $q$ tương đương nhau, nên $\lnot p$ và $\lnot q$ cũng tương đương nhau. Sử dụng luật De Morgan nhiều lần để đẩy toán tử $\lnot$ trong $p$ và $q$ sâu nhất có thể, khi đó, toán tử $\land$ sẽ thay bằng $\lor$, $\textbf{T}$ thay bằng $\textbf{F}$, và ngược lại. Từ đó, mệnh đề $\lnot p$ và $\lnot q$ sẽ giống với $p^*$ và $q^*$, ngoại trừ những mệnh đề đơn vị $p_i$ được thay bằng mệnh đề phủ định của nó. Khi đó $p^*$ và $q^*$ tương đương nhau, vì $\lnot p$ và $\lnot q$ tương đương nhau. \\\ \\\
\textbf{Bài tập 1.3.40: }Find a compound proposition involving the propositional variables p, q, and r that is true when p and q are true and r is false, but is false otherwise. \\\ \\\
\textbf{Lời giải: } \\\ \\\
Xét mệnh đề $p \land q \land \lnot r.$ \\\
Ta có bảng giá trị sau: \\\  \\\
\begin{tabular}{|c|c|c|c|c|}
\hline 
p & q & r & $\lnot r$ & $p \land q \land \lnot r$\\ 
\hline 
1 & 1 & 0 & 1 & 1\\ 
\hline 
0 & 0 & 1 & 0 & 0\\ 
\hline 
\end{tabular} \\\ \\\
Theo bảng giá trị này, ta suy ra được mệnh đề trên thỏa mãn yêu cầu đề bài. \\\ \\\
\clearpage

\section{DS\_predicatelogic.pdf}
\subsection{Bài tập bắt buộc}
\subsubsection{Bài tập 3}
\textbf{Đề bài:} 
\\\ \\\
\textbf{Lời giải:} \\\ \\\
\clearpage
\subsubsection{Bài tập 4}
\textbf{Đề bài:} 
\\\ \\\
\textbf{Lời giải:} \\\ \\\
\clearpage
\subsubsection{Bài tập 5}
\textbf{Đề bài:} 
\\\ \\\
\textbf{Lời giải:} \\\ \\\
\clearpage
\subsubsection{Bài tập 6}
\textbf{Đề bài:} 
\\\ \\\
\textbf{Lời giải:} \\\ \\\
\clearpage
\subsubsection{Bài tập 7}
\textbf{Đề bài:} 
\\\ \\\
\textbf{Lời giải:} \\\ \\\
\clearpage
\subsubsection{Bài tập 8}
\textbf{Đề bài:} 
\\\ \\\
\textbf{Lời giải:} \\\ \\\
\clearpage
\subsubsection{Bài tập 9}
\textbf{Đề bài:} 
\\\ \\\
\textbf{Lời giải:} \\\ \\\
\clearpage
\subsubsection{Bài tập 10}
\textbf{Đề bài:} 
\\\ \\\
\textbf{Lời giải:} \\\ \\\
\clearpage
\subsubsection{Bài tập 11}
\textbf{Đề bài:} 
\\\ \\\
\textbf{Lời giải:} \\\ \\\
\clearpage
\subsubsection{Bài tập 12}
\textbf{Đề bài:} 
\\\ \\\
\textbf{Lời giải:} \\\ \\\
\clearpage
\subsubsection{Bài tập 13}
\textbf{Đề bài:} 
\\\ \\\
\textbf{Lời giải:} \\\ \\\
\clearpage
\subsubsection{Bài tập 14}
\textbf{Đề bài:} 
\\\ \\\
\textbf{Lời giải:} \\\ \\\
\clearpage
\subsubsection{Bài tập 15}
\textbf{Đề bài:} Use rules of inference to show that if $\forall x (P(x) \lor Q(x)), \forall x (\lnot Q(x) \lor S(x)), \forall x (R(x) \implies \lnot S(x))$ and $ \exists x \lnot P(x)$ are true, then $\exists x \lnot R(x) is true$.
\\\ \\\
\textbf{Lời giải:} \\\ \\\
\begin{enumerate}
	\item $\exists x \lnot P(x)$ \hfill Premise\\
	\item $\lnot P(a)$\hfill Existentail Instantiation from (1)\\
	\item $ \forall x (P(x) \lor Q(x))$\hfill Premise\\
	\item $ P(a) \lor Q(a)$\hfill Universal Instantiation from (3)\\
	\item $ Q(a)$\hfill Disjunctive syllogism from (2) and (4)\\
	\item $\forall x (\lnot Q(x) \lor S(x))$\hfill Premise\\
	\item $ Q(a) \lor S(a)$\hfill Universal Instantiation from (6)\\
	\item $ S(a)$\hfill Disjunctive syllogism from (5) and (7)\\
	\item $ \forall x (R(x) \implies \lnot S(x))$\hfill Premise \\
	\item $ \lnot R(a) \implies S(a)$\hfill Universal Instantiation from (9)\\
	\item $ \lnot R(a)$\hfill Moduns tollens from (7) and (9)\\
\end{enumerate}
\clearpage
\subsubsection{Bài tập 16}
\textbf{Đề bài:} Use a direct proof to show that the sum of two odd integers is even. 
\\\ \\\
\textbf{Lời giải:} \\\ \\\
if $a$,$b$ are two odd integer, then $a+b$ is even\\
Assume that $a$ and $b$ are odd. By the definition, $a=2m+1$ and $b+2n+1$, ($m$,$n$ $\in Z$)\\
$a+b=2m+2n+2=2(m+n+1)$ is an even number\\
\clearpage
\subsubsection{Bài tập 17}
\textbf{Đề bài:} Use a direct proof to show that the product of two odd numbers is odd.
\\\ \\\
\textbf{Lời giải:} \\\ \\\
if $a$,$b$ are odd number, then the product of a,b is odd\\
Assume that $a$ and $b$ are odd. By the definition, $a=2m+1$ and $b+2n+1$, ($m$,$n$ $\in Z$)\\
$ab=(2m+1)(2n+1)=4mn+2m+2n+1=2(2mn+m+n)+1$ is an odd number\\
\clearpage
\subsubsection{Bài tập 18} 
\textbf{Đề bài:} Use a direct proof to show that every odd integer is the difference of two squares.
\\\ \\\
\textbf{Lời giải:} \\\ \\\
if $a$ is odd then $a$ is the result of the difference of the two squares
Assume that $a$ is odd. By the definition $a=2n+1$ $n \in Z$	
$a=2n+1=n^{2}+2n+1-n^{2}=(n+1)^{2}-n^{2}$\\
\clearpage
\subsubsection{Bài tập 19}
\textbf{Đề bài:} Proof that if n + m and n + p are even integers, where m; n; p are integers, then m + p is even. What
kind of proof did you use?
\\\ \\\
\textbf{Lời giải:} \\\ \\\
Using direct proof
Assume that $n+m$,$n+p$ are even. By the definition $n+m=2a+1$,$n+p=2b+1$ $a,b \in Z$
$n+m+n+p=2a+2b+2=2(a+b+1)$\\
$\equiv m+p=2(a+b+1)-2n$\\
$\equiv m+p=2(a+b+1-n)$\\		
\clearpage
\subsubsection{Bài tập 20} Prove that the sum of two rational numbers is rational.
\textbf{Đề bài:} 
\\\ \\\
\textbf{Lời giải:} \\\ \\\
if $a$,$b$ are rational, then $a+b$ is ratioaal\\
Assume $a$,$b$ is rational. By the definition, $a=\frac{m}{n}$,$b=\frac{p}{q}$\\
$a+b=\frac{m}{n}+\frac{p}{q}=\frac{mq+np}{p+q}$\\
\clearpage
\subsubsection{Bài tập 21}
\textbf{Đề bài:} Use a proof by contradiction to prove that the sum of an irrational number and a rational number is irrational.
\\\ \\\
\textbf{Lời giải:} \\\ \\\
if $a$ is an irrational number and $b$ is a rational number then $a+b$ is an irrational number
Assume that $a$ is irrational number, $b$ is a rational number.\\
Proposition $a+b$ is rational, $(a+b)-b=a$ is rational\\
We assume that $a$ is irrational. In proposition, we proof that a is rational. That means proposition contradict with the assumption. So $a+b$ is irrational number\\ 
\clearpage
\subsubsection{Bài tập 22}
\textbf{Đề bài: }Prove that if $x$ is irrational, then $1/x$ is irrational. \\\ \\\
\textbf{Lời giải:} \\\ \\\
Ta chứng minh bài toán bằng phương pháp phản chứng. Giả sử rằng tồn tại một số vô tỉ $x$ sao cho $1/x$ là số hữu tỉ. Vì $1/x$ là một số hữu tỉ nên tồn tại hai số nguyên $a,b (b \neq 0)$ sao cho :$\frac{1}{x} = \frac{a}{b}.$ Tương đương với $x = \frac{b}{a}$. Suy ra $x$ là số hữu tỉ (mâu thuẫn với $x$ là số vô tỉ). \\\
Vậy ta có điều phải chứng minh.
\clearpage
\subsubsection{Bài tập 23}
\textbf{Đề bài: } Use a proof by contraposition to show that if $x + y \geq 2$, where $x$ and $y$ are real numbers, then $x \geq 1$ or $y \geq 1$.\\\ \\\
\textbf{Lời giải:} \\\ \\\
Ta sẽ chứng minh rằng, nếu $x < 1$ và $y < 1$ thì $x+y< 2$. \\\
Thật vậy, ta có $x < 1$ và $y < 1 \Leftrightarrow x+y < 1+1 = 2$. \\\
Phản đảo lại, ta được: nếu $x+y \geq 2$ thì $x \geq 1$ hoặc $y \geq 1$. \\\
Ta có điều phải chứng minh.

\clearpage
\subsubsection{Bài tập 24}
\textbf{Đề bài: } Show that if $n$ is an integer and $n^3 + 2015$ is odd, then $n$ is even using \\\ \\\
a) a proof by contraposition. \\\
b) a proof by contradiction.\\\ \\\
\textbf{Lời giải:} \\\ \\\
Xét $n$ là một số nguyên \\\
a) Ta sẽ chứng minh rằng nếu $n$ là số lẻ thì $n^3 + 2015$ chẵn. \\\
Thật vậy, nếu $n$ là số lẻ thì tồn tại số nguyên $k$ sao cho $n = 2k+1$. Khi đó: $n^3+2015 = (2k+1)^3+2015=8k^3 + 12k^2+6k + 2016$ là một số chẵn.\\\
Phản đảo lại, ta được: nếu $n^3+2015$ là số lẻ thì $n$ là số chẵn. \\\
Ta có điều phải chứng minh. \\\ \\\
b) Ta sẽ đi chứng minh phản chứng bài toán. Giả sử tồn tại một số $n$ lẻ sao cho $n^3+2015$ lẻ. Vì $n$ là số lẻ nên tồn tại số nguyên $k$ sao cho $n = 2k+1$. Khi đó: $n^3+2015 = (2k+1)^3+2015=8k^3 + 12k^2+6k + 2016$ là một số chẵn (mâu thuẫn với dữ kiện $n^3 + 2015$ lẻ). \\\
Ta có điều phải chứng minh.

\clearpage
\subsubsection{Bài tập 25}
\textbf{Đề bài: } Prove that if $n$ is an integer and $3n + 2$ is even, then $n$ is even using \\\ \\\
a) a proof by contraposition. \\\
b) a proof by contradiction.\\\ \\\
\textbf{Lời giải:} \\\ \\\
Xét $n$ là số một số nguyên \\\
a) Ta sẽ chứng minh rằng nếu $n$ là số lẻ thì $3n+2$ lẻ. \\\
Thật vậy, nếu $n$ là số lẻ thì tồn tại số nguyên $k$ sao cho $n = 2k+1$. Khi đó: $3n+2 = 3(2k+1)+2 = 6k + 5$ là một số lẻ.\\\
Phản đảo lại, ta được: nếu $3n+2$ là số chẵn thì $n$ là số chẵn. \\\
Ta có điều phải chứng minh. \\\ \\\
b) Ta sẽ đi chứng minh phản chứng bài toán. Giả sử tồn tại số $n$ lẻ sao cho $3n+2$ chẵn. Vì $n$ là số lẻ nên tồn tại số nguyên $k$ sao cho $n = 2k+1$. Khi đó: $3n+2 = 3(2k+1)+2 = 6k + 5$ là một số lẻ (mâu thuẫn với dữ kiện $3n+2$ chẵn). \\\
Ta có điều phải chứng minh.
\\\ \\\
\clearpage
\subsubsection{Bài tập 26}
\textbf{Đề bài: } Prove that if $n$ is a positive integer, then $n$ is odd if and only if $5n + 6$ is odd.\\\ \\\
\textbf{Lời giải:} \\\ \\\
Xét $n$ là số nguyên dương. \\\
Ta đi chứng minh hai chiều như sau:
\begin{enumerate}
\item Nếu $n$ lẻ thì $5n+6$ lẻ.
\item Nếu $5n+6$ lẻ thì $n$ nguyên lẻ.
\end{enumerate}
* Chiều thứ nhất: \\\
Vì $n$ lẻ nên tồn tại số nguyên $k$ sao cho $n=2k+1$. Khi đó $5n+6=5(2k+1)+6=10k+11$ là một số lẻ. Vậy chiều này được chứng minh. \\\ \\\
* Chiều thứ hai: \\\
Ta chứng minh rằng nếu $n$ chẵn thì $5n+6$ chẵn. Vì $n$ chẵn nên tồn tại số nguyên $k$ sao cho $n=2k$. Khi đó $5n+6=5.2k+6=10k+6$ là một số chẵn. \\\
Phản đảo lại, ta được: nếu $5n+6$ là số lẻ thì $n$ là số lẻ. Vậy chiều này được chứng minh. \\\ \\\
Ta có điều phải chứng minh.

\clearpage
\subsubsection{Bài tập 27}
\textbf{Đề bài: }Show that these statements about the integer $x$ are equivalent: (i) $3x + 2$ is even, (ii) $x + 5$ is odd,
(iii) $x^2$ is even.\\ \\\
\textbf{Lời giải:} \\\ \\\
Xét $n$ là số nguyên. \\\
Ta sẽ đi chứng minh 2 vị từ sau:
\begin{enumerate}
\item $3x + 2$ chẵn khi và chỉ khi $x + 5$ lẻ.
\item $x+5$ lẻ khi và chỉ khi $x^2$ chẵn.
\end{enumerate}
* Vị từ 1:
\begin{enumerate}
\item Nếu $3x+2$ chẵn thì $x+5$ lẻ. \\\
Ta đi chứng minh rằng nếu $x+5$ chẵn thì $3x+2$ lẻ. \\\
Thật vậy, vì $x+5$ chẵn nên tồn tại số nguyên $k$ thỏa mãn: $x+5 = 2k$. Khi đó ta có $3x+2=3(x+5)-13=6k-13$ là một số lẻ.  \\\
Phản đảo lại, ta được nếu $3x+2$ chẵn thì $x+2$ lẻ.
\item Nếu $x+5$ lẻ thì $3x+2$ chẵn. \\\
Vì $x+5$ lẻ nên tồn tại số nguyên $k$ thỏa mãn: $x+5 = 2k+1$. Khi đó ta có $3x+2=3(x+5)-13=3(2k+1)-13 = 6k-10$ là một số chẵn.  \\\
\end{enumerate}
Vậy ta chứng minh được vị từ 1. \\\ \\\
* Vị từ 2:
\begin{enumerate}
\item Nếu $x^2$ chẵn thì $x+5$ lẻ. \\\
Ta đi chứng minh rằng nếu $x+5$ chẵn thì $x^2$ lẻ. \\\
Thật vậy, vì $x+5$ chẵn nên tồn tại số nguyên $k$ thỏa mãn: $x+5 = 2k$. Khi đó ta có $x^2 = (x+5-5)^2=(2k-5)^2 = 4k^2-20k+25$ là một số lẻ.  \\\
Phản đảo lại, ta được nếu $x^2$ chẵn thì $x+5$ lẻ.
\item Nếu $x+5$ lẻ thì $x^2$ chẵn. \\\
Vì $x+5$ lẻ nên tồn tại số nguyên $k$ thỏa mãn: $x+5 = 2k+1$. Khi đó ta có $x^2 = (x+5-5)^2=(2k-4)^2=4(k-2)^2$ là một số chẵn.  \\\
\end{enumerate}
Vậy ta chứng minh được vị từ 2. \\\
Vì (i), (ii), (iii) tương đương nhau nên ta có điều phải chứng minh.
\clearpage
\subsubsection{Bài tập 28}
\textbf{Đề bài: } Prove that if $n$ is an integer, these four statements are equivalent: (i) $n$ is even, (ii) $n + 1$ is odd, (iii) $3n + 1$ is odd, (iv) $3n$ is even.\\\ \\\
\textbf{Lời giải:} \\\ \\\
Xét $n$ là số nguyên. \\\
Ta sẽ chứng minh 2 vị từ sau:
\begin{enumerate}
\item $n$ chẵn khi và chỉ khi $n+1$ lẻ.
\item $n$ chẵn khi và chỉ khi $3n$ chẵn.
\end{enumerate}
* Vị từ 1: 
\begin{enumerate}
\item Nếu $n$ chẵn thì $n+1$ lẻ. \\\
Vì $n$ chẵn nên tồn tại số nguyên $k$ thỏa: $n=2k$. Khi đó $n+1=2k+1$ là một số lẻ.
\item Nếu $n+1$ lẻ thì $n$ chẵn. \\\
Vì $n+1$ lẻ nên tồn tại số nguyên $k$ thỏa: $n+1=2k+1$. Khi đó $n=n+1-1=2k+1-1=2k$ là một số chẵn.
\end{enumerate}
Vậy ta chứng minh được vị từ 1. \\\ \\\
* Vị từ 2:
\begin{enumerate}
\item Nếu $n$ chẵn thì $3n$ chẵn \\\
Vì $n$ chẵn nên tồn tại số nguyên $k$ sao cho $n=2k$. Khi đó $3n=3.2k=6k$ là một số chẵn.
\item Nếu $3n$ chẵn thì $n$ chẵn \\\
Ta đi chứng minh rằng nếu $n$ lẻ thì $3n$ lẻ. \\\
Vì $n$ lẻ nên tồn tại số nguyên $k$ sao cho $n=2k+1$. Khi đó $3n=3.(2k+1) = 6k+3$ là một số lẻ. \\\
Phản đảo lại, ta được: Nếu $3n$ chẵn thì $n$ chẵn.
\end{enumerate}
Vậy ta chứng minh được vị từ 2. \\\
Ta có $3n+1$ lẻ $\equiv 3n$ chẵn $\equiv n$ chẵn $\equiv n+1$ lẻ. \\\
Vậy ta có điều phải chứng minh. 

\clearpage
\subsubsection{Bài tập 29}
\textbf{Đề bài:} 
\\\ \\\
\textbf{Lời giải:} \\\ \\\
\clearpage
\subsubsection{Bài tập 30}
\textbf{Đề bài:} 
\\\ \\\
\textbf{Lời giải:} \\\ \\\
\clearpage
\subsubsection{Bài tập 31}
\textbf{Đề bài:} 
\\\ \\\
\textbf{Lời giải:} \\\ \\\
\clearpage
\subsubsection{Bài tập 32}
\textbf{Đề bài:} 
\\\ \\\
\textbf{Lời giải:} \\\ \\\
\clearpage
\subsubsection{Bài tập 33}
\textbf{Đề bài:} 
\\\ \\\
\textbf{Lời giải:} \\\ \\\
\clearpage
\subsubsection{Bài tập 34}
\textbf{Đề bài:} 
\\\ \\\
\textbf{Lời giải:} \\\ \\\
\clearpage
\subsubsection{Bài tập 35}
\textbf{Đề bài:} 
\\\ \\\
\textbf{Lời giải:} \\\ \\\
\clearpage
\clearpage

\section{New\_Homework02a\_Predicate\_Logic.pdf}
\subsection{Bài tập bắt buộc}
\subsubsection{Bài tập 1}
\textbf{Đề bài:} 
\\\ \\\
\textbf{Lời giải:} \\\ \\\
\clearpage
\subsubsection{Bài tập 2}
\textbf{Đề bài:} 
\\\ \\\
\textbf{Lời giải:} \\\ \\\
\clearpage
\subsubsection{Bài tập 3}
\textbf{Đề bài:} Let P(x; y), Q(x; y), and R(x) be propositional functions. Use logical equivalences to show that $\lnot \forall x ((\exists y (P(x,y) \implies Q(x,y))) \lor R(x))$ and $\equiv \exists x (\lnot R(x) \land \forall y (\lnot Q(x,y) \land P(x,y)))$ are equipvalent.
\\\ \\\
\textbf{Lời giải:} \\\ \\\
$\lnot \forall x ((\exists y (P(x,y) \implies Q(x,y))) \lor R(x))$\\
$\equiv \lnot \forall x ((\exists y (\lnot P(x,y) \lor Q(x,y))) \lor R(x))$\\
$\equiv \exists x ((\forall y (P(x,y) \land \lnot Q(x,y))) \land \lnot R(x)$\\
$\equiv \exists x (\lnot R(x) \land \forall y (\lnot Q(x,y) \land P(x,y)))$\\
\clearpage
\subsubsection{Bài tập 4}
\textbf{Đề bài:} Use rules of inference to show that if $p \land q$, $r \lor s$, and $p \rightarrow \lnot r$, then $s$ is true. \\\ \\\
\textbf{Lời giải:} \\\ \\\
We have:
\begin{enumerate}
\item $p \land q$ (Premise).
\item $p$ (Simplification from (1)).
\item $p \rightarrow \lnot r$ (Premise).
\item $\lnot r$ (Modus pones using (2) and (3)).
\item $r \lor s$ (Premise).
\item $s$ (Disjunctive syllogism using (4) and (5)).
\end{enumerate}
Q.E.D

\clearpage
\subsubsection{Bài tập 5}
\textbf{Đề bài:} 
\\\ \\\
\textbf{Lời giải:} \\\ \\\
\clearpage

\subsection{Bonus}
\textbf{Bài tập 1,4.44:} Determine whether $\forall x (P(x) \Leftrightarrow Q(x))$ and $\forall x P(x) \Leftrightarrow \forall x Q(x)$ are logically equipvalent. Justify your answer.
Assume that $\forall x P(x) \Leftrightarrow \forall x Q(x)$ is true. So for all x, Q(x) is true and P(x) is true. Therefore,  $\forall x (P(x) \Leftrightarrow Q(x))$.\\
Assume that  $\forall x (P(x) \Leftrightarrow Q(x))$ is true. So for all x P(x), Q(x) have the same value. Therefore $\forall x P(x) \Leftrightarrow \forall x Q(x)$ is true.\\
Therefore, $\forall x (P(x) \Leftrightarrow Q(x)) \equiv \forall x P(x) \Leftrightarrow \forall x Q(x)$
\textbf{Bài tập 1.4.48:} Establish these logical equivalences, where x does not occur as a free variable in A. Assume that the domain is nonempty.
\begin{enumerate}
	\item[a)] $\forall x (A \implies P(x) \equiv A \implies \forall x P(x)$
	If A is false, both sides are  always true, If A is true, the value of both sides depend on P(x). Therefore, the two sides are logically equivalent.
	\item[b)] $\exists x (A \implies P(x)) \equiv A \implies \exists x P(x)$
	If A is false, both sides are  always true, If A is true, the value of both sides depend on P(x). Therefore, the two sides are logically equivalent.
\end{enumerate}
\textbf{Bài tập 1.4.62:} Let P (x), Q(x), R(x), and S(x) be the statements “x is a duck,” “x is one of my poultry,” “x is an officer,” and “x is willing to waltz,” respectively. Express each of these statements using quantifiers; logical connectives; and P (x), Q(x), R(x), and S(x).
	\begin{enumerate}
	\item[a)] No duck are willing to waltz.
	$\forall x (P(x) \implies \lnot S(x)$
	\item[b)] No officers ever discline to waltz.
	$\forall x (R(x) \implies S(x))$
	\item[c)] All my poultry are ducks.
	$\forall x (Q(x) \implies P(x))$
	\item[d)] My poulrtry are not officers
	$\forall x (Q(x) \implies \lnot R(x))$
	\item[e)] d is follow a,b and c.
	\subitem1) $\forall x (P(x) \implies \lnot S(x)$ \hfill Premise
	\subitem2)	 $\forall x (R(x) \implies S(x))$ \hfill Premise
	\subitem3) $\forall x (Q(x) \implies P(x))$ \hfill Premise
	\subitem4) $ P(a) \implies \lnot S(a)$ \hfill Universal Instantiation from (1)
	\subitem5) $ R(a) \implies S(a)$ \hfill Universal Instantiation from (2)
	\subitem6)$ Q(a) \implies P(a)$ \hfill Universal Instantiation from (3)
	\subitem7) $ Q(a) \implies \lnot S(a)$ \hfill Hypothegical Syllogism from (4) and (6)
	\subitem8) $ \lnot R(a) \lor  S(a)$ \hfill Logical Equivalence from (5)
	\subitem9) $ \lnot S(a) \implies \lnot R(a)$ \hfill Logical Equivalence from (8)
	\subitem10) $ Q(a) \implies \lnot R(a)$ \hfill Hypothegical Syllogism from (7) and (9)
	\subitem11) $\forall x Q(x) \implies \lnot R(x)$ \hfill Universal generlization from (10)
\textbf{Bài tập 1.5.1:} Translate these statements into English, where the domain for each variable consists of all real numbers.
\begin{enumerate}
	\item[a)] $\forall x \exists y (x<y)$
	For all x, exisrs y that x is greater than y.\\
	\item[b)] $\forall x \forall y (((x \geq 0) \land (y \geq 0)) \implies (xy \geq 0))$
	For all x, For all y that if x  is greater than or equal to 0 and y is greater than or equal to 0, xy is greater than or equal to 0.\\
	\item[c)] $\forall x \forall y \exists z (xy=z)$
	For all x, For all y, Exists z that xy equal to z.\\
\end{enumerate}
\textbf{Bài tập 1.5.10} Let F (x, y) be the statement “x can fool y,” where the domain consists of all people in the world. Use quantifiers to express each of these statements.
\begin{enumerate}
	\item[a)] Everybody can fool Fred.
	$\forall x F(x,Fred)$
	\item[b)] Evelyn can fool everybody
	$\forall x F(Evelyn,x)$
	\item[c)] Everybody can fool sombody.
	$\forall x \exists y F(x,y)$
	\item[d)] There is no one who can fool everybody.
	$\forall x, \exists y \lnot F(x,y)$
	\item[e)] Everyone can be fooled by somebody.
	$\exists y \forall x F(y,x)$
	\item[f)] No one can fool both Fred and Jerry.
	$\forall x (F(x,Fred) \oplus F(x,Jerry))$
	\item[g)] Nancy can fool exactly two people.
	$\exists x,y (x \ne y) \implies (F(Nancy,x) \land F(Nancy,y))$
\end{enumerate}
\textbf{Bài tập 1.5.18: }Express each of these system specifications using predicates, quantifiers, and logical connectives, if necessary.
\begin{enumerate}[a)]
\item At least one console must be accessible during every
fault condition.
\item The e-mail address of every user can be retrieved
whenever the archive contains at least one message sent by every user on the system.
\item For every security breach there is at least one mechanism that can detect that breach if and only if there is a process that has not been compromised.
\item There are at least two paths connecting every two distinct endpoints on the network.
\item No one knows the password of every user on the system except for the system administrator, who knows
all passwords.
\end{enumerate}
\textbf{Lời giải:}
\begin{enumerate}[a)]
\item $P(x)$: Console $x$ can be accessible. \\\
$Q(x)$: $x$ is a fault condition. \\\
Câu trên được viết lại thành: $\forall x Q(x) (\exists y P(y)).$
\item $P(x)$: The e-mail address of $x$ can be retrieved. \\\
$Q(x)$: The archive contains messages sent by $x$. \\\
Câu trên được viết lại thành: $(\forall x Q(x))\rightarrow (\forall y P(y)).$
\item $P(x)$: $x$ is security breach. \\\
$Q(x,y)$: $x$ is a mechanism that can detech security breach $y$. \\\
$R(x)$: $x$ is the process that has not been compromised. \\\
Câu trên được viết lại thành: $\forall x P(x) ((\exists yQ(y,x)) \leftrightarrow (\exists zR(z)))$.
\item $P(x,y,z)$: $x$ and $y$ are connected by $z$. \\\
Câu trên được viết lại thành: $\forall x \forall y ((x \neq y) \rightarrow (\exists a \exists b ((a \neq b) \rightarrow (P(x,y,a) \land P(x,y,b))))).$
\item $P(x,y)$: $x$ knows the password of $y$. \\\
$Q(x):$ $x$ is the system administrator. \\\
Câu trên được viết lại thành: $(\forall x (\lnot Q(x))(\exists y (\lnot P(x,y))) \land ((\forall xQ(x)) \rightarrow (\forall y P(x,y))).$
\end{enumerate} 
\textbf{Câu 1.5.19: }Express each of these statements using mathematical and logical operators, predicates, and quantifiers, where the domain consists of all integers.
\begin{enumerate}[a)]
\item The sum of two negative integers is negative.
\item The difference of two positive integers is not necessarily positive.
\item The sum of the squares of two integers is greater than or equal to the square of their sum.
\item The absolute value of the product of two integers is the product of their absolute values.
\end{enumerate}
\textbf{Lời giải: }
\begin{enumerate}[a)]
\item $\forall x\forall y (x < 0 \land y < 0) \rightarrow (x+y < 0)$.
\item $\exists x \exists y (x > 0 \land y > 0) \rightarrow (x-y \leq 0)$.
\item $\forall x \forall y(x^2+y^2 \geq (x+y)^2)$.
\item $\forall x \forall y(|x.y| = |x|.|y|)$.
\end{enumerate}
\textbf{Câu 1.5.20: }Express each of these statements using predicates, quantifiers, logical connectives, and mathematical operators where the domain consists of all integers.
\begin{enumerate}[a)]
\item The product of two negative integers is positive.
\item The average of two positive integers is positive.
\item The difference of two negative integers is not necessarily negative.
\item The absolute value of the sum of two integers does not exceed the sum of the absolute values of these integers.
\end{enumerate}
\textbf{Lời giải: }
\begin{enumerate}[a)]
\item $\forall x \forall y (x < 0 \land y < 0) \rightarrow (xy > 0)$.
\item $\forall x \forall y (x > 0 \land y > 0) \rightarrow (\frac{x+y}{2} > 0)$.
\item $\exists x \exists y (x < 0 \land y < 0) \rightarrow (x-y \geq 0)$.
\item $\forall x \forall y (|x+y| \leq |x|+|y|)$.
\end{enumerate}
\textbf{Câu 1.5.21: }Use predicates, quantifiers, logical connectives, and mathematical operators to express the statement that every positive integer is the sum of the squares of four integers. \\\ \\\
\textbf{Lời giải: } \\\ \\\
Domain: Tập số nguyên. \\\
Ta viết lại câu trên thành: \\\
$\forall x (x > 0) \rightarrow (\exists a,b,c,d(x = a^2+b^2+c^2+d^2)).$ \\\ \\\
\textbf{Câu 1.5.22: }Use predicates, quantifiers, logical connectives, and mathematical operators to express the statement that there is a positive integer that is not the sum of three squares. \\\ \\\
\textbf{Lời giải:} \\\ \\\
Domain: Tập số nguyên. \\\
Ta viết lại câu trên thành: \\\
$\exists x (x > 0) \rightarrow (\forall a,b,c(x \neq a^2+b^2+c^2)).$\\\ \\\
\textbf{Câu 1.5.23: }Express each of these mathematical statements using predicates, quantifiers, logical connectives, and mathematical operators.
\begin{enumerate}[a)]
\item The product of two negative real numbers is positive.
\item The difference of a real number and itself is zero.
\item Every positive real number has exactly two square
roots.
\item A negative real number does not have a square root
that is a real number.
\end{enumerate}
\textbf{Lời giải: } \\\ \\\
Domain: Tập số thực.
\begin{enumerate}[a)]
\item $\forall x \forall y(x < 0 \land y < 0) \rightarrow (xy > 0).$
\item $\forall x(x - x = 0).$
\item $\forall x (x>0)\rightarrow ((\exists a \exists b ((a \neq b) \land (a^2 = b^2 = x))) \land (\forall c ((c \neq a) \land (c \neq b))\rightarrow(c^2 \neq x)).$
\item $\forall x (x < 0) \rightarrow (\forall y (y^2 \neq x)).$
\end{enumerate}

\clearpage

\section{New\_Homework02b\_Proving\_methods.pdf}
\subsection{Bài tập bắt buộc}
\subsubsection{Bài tập 1}
\textbf{Đề bài:} 
\\\ \\\
\textbf{Lời giải:} \\\ \\\
\clearpage
\subsubsection{Bài tập 2}
\textbf{Đề bài:} 
\\\ \\\
\textbf{Lời giải:} \\\ \\\
\clearpage
\subsubsection{Bài tập 3}
\textbf{Đề bài:} Prove that $\sqrt{35}$ is irrational
\\\ \\\
\textbf{Lời giải:} \\\ \\\
Assume that $\sqrt{35}$ is rational. if so, $\exists a,b (a,b \in Z) \sqrt{35}=\frac{a}{b}$. a,b is the smallest such as possible integer.\\
$\frac{a^{2}}{b^{2}}=35 \equiv a^{2}=35b^{2}$\\
The right hand has factors of 5 and 7, so $a^{2}$ must be divisible by 5 and 7. By the unique prime factorization theorem , a must be divisible by 5 and 7 too. So $a=35k (k \in N)$ then $a^{2}=(35k)^{2}$\\
$\equiv (35k)^{2}=35(b)^{2}$\\
$\equiv 35(k)^{2}=(b)^{2}$\\
$\equiv (\frac{b}{k})^{2}=35$\\
$\equiv \frac{b}{k}=\sqrt{35}$\\
Because $\sqrt{35}$>1, b>k, so it is contradicting our assumption. Therefore, $\sqrt{35}$ is irrational. 
\clearpage
\subsubsection{Bài tập 4}
\textbf{Đề bài:} In the country of Togliristan (where Knights, Knaves, and Togglers live), Togglers will alternate between telling the truth and lying (no matter what other people say). You meet two people, A and B. They say, in order: \\\

A : B is a Knave. 

B : A is a Knave.

A : B is a Knight.

B : A is a Toggler.\\\ \\\
Determine what types of people A and B are. \\\ \\\
\textbf{Lời giải:} \\\ \\\
Vì A,B đều có hai câu nói khác nhau nên A và B không thể là Knight được. Ta xét hai trường hợp: 
\begin{enumerate}
\item A là Knave \\\
Vì A là Knave nên A luôn nói dối, hay B không thể là Knave hay Knight. Khi đó B sẽ là Toggler.
\item A là Toggler \\\
Nếu B là Knave thì B luôn nói dối, hay A không thể là Knave hay Toggler (mâu thuẫn). Do đó B là Toggler.
\end{enumerate}
Vậy (A,B) chỉ có thể là (Knave, Toggler) và (Toggler, Toggler).

\clearpage
\subsubsection{Bài tập 5}
\textbf{Đề bài:} 
\\\ \\\
\textbf{Lời giải:} \\\ \\\
\clearpage
\subsection{Bonus}
\textbf{Bài tập 1.8.1:} Prove that $n{2}+1 \geq 2{n}$ when n is a positive integer with $1\geq n \geq 4$\\
With  $1\geq n \geq 4$, we have n=1, n=2, n=3, n=4\\
For n=1, we have $n^{2}+1=1^{2}+1=2$ and $2^{n}=2^{1}=2$\\
For n=2, we have $n^{2}+1=2^{2}+1=5$ and $2^{n}=2^{2}=4$\\
For n=3, we have $n^{2}+1=3^{2}+1=10$ and $2^{n}=2^{3}=8$\\
For n=4, we have $n^{2}+1=4^{2}+1=17$ and $2^{n}=2^{4}=16$\\
In each of these four cases we see that $n^{2}+1 \geq 2^{n}$ with $1\geq n \geq 4$\\

\textbf{Bài tập 1.8.2:}  Prove that there are no positive perfect cubes less than 1000 that are the sum of the cubes of two positive integers.
Assume that $0\lneq a^{3} \le 1000$ is greater than or equal to the sum of the cubes of b and c with a, b, c $\in$ Z and $a\gneq b \geq c$. Therefore the value of a is from 3 to 9.\\
For a=9, we have $a^{3}=9^{3}=729$. So that the highest value of b and c are 8 and 6. Then $b^{3} + c^{3}=8^{3} + 6^{3}=728$.\\
For a=8, we have $ a^{3}=8^{3}=512$. So that the highest value of b and c are 7 and 5. Then $b^{3} + c^{3}=7^{3} + 5^{3}=468$.\\
For a=7, we have $a^{3}=7^{3}=343$. So that the highest value of b and c are 6 and 5. Then $b^{3} + c^{3}=6^{3} + 5^{3}=341$.\\
For a=6, we have $a^{3}=6^{3}=216$. So that the highest value of b and c are 5 and 4. Then $b^{3} + c^{3}=5^{3} + 4^{3}=189$.\\
For a=5, we have $a^{3}=5^{3}=125$. So that the highest value of b and c are 4 and 3. Then $b^{3} + c^{3}=4^{3} + 3^{3}=91$.\\
For a=4, we have $a^{3}=4^{3}=64$. So that the highest value of b and c are 3 and 3. Then $b^{3} + c^{3}=3^{3} + 3^{3}=54$.\\
For a=3, we have $a^{3}=3^{3}=27$. So that the highest value of b and c are 2 and 2. Then $b^{3} + c^{3}=2^{3} + 2^{3}=16$.\\
In each of these cases, we see that for all positive perfect cubes less than 1000 that are not the sum of the cubes of two positive integers.\\

\textbf{Bài tập 1.8.3:}  Prove that if x and y are real numbers, then max(x, y) + min(x, y) = x + y. [Hint: Use a proof by cases, with the two cases corresponding to x ≥ y and x < y, respectively.]\\
Assume that $x \ge y$. So that max(x, y) = x, min(x, y) = y. Therefore, max(x, y) + min(x, y) = x + y.\\
Assume that x < y. So that max(x, y) = y, min(x, y) = x. Therefore, max(x, y) + min(x, y) = y + x = x + y.\\

\textbf{Bài tập 1.8.4:} Use a proof by cases to show that min(a, min(b, c)) =
min(min(a, b), c) whenever a, b, and c are real numbers.\\
Assume that a > b > c. So that min(min(a, b), c) = min(b, c) = c and min(a, min(b, c)) = min(a, c) = c. Therefore, min(min(a, b), c) = min(a, min(b, c)).\\
Assume that a > c > b. So that min(min(a, b), c) = min(b, c) = b and min(a, min(b, c)) = min(a, b) = b. Therefore, min(min(a, b), c) = min(a,min(b, c)).\\
Assume that b > a > c. So that min(min(a, b), c) = min(a, c) = c and min(a, min(b,  c)) = min(a, c) = c. Therefore, min(min(a, b), c) = min(a,min(b, c)).\\
Assume that b > c > a. So that min(min(a, b), c) = min(a, c) = a and min(a, min(b, c)) = min(a, c) = a. Therefore, min(min(a, b), c) = min(a,min(b, c)).\\
Assume that c > a > b. So that min(min(a, b), c) = min(b, c) = b and min(a, min(b, c)) = min(a, b) = b. Therefore, min(min(a, b), c)= min(a, min(b, c)).\\
Assume that c > b > a. So that min(min(a, b), c) = min(a, c) = a and min(a, min(b, c)) = min(a, b) = a. Therefore, min(min(a, b), c)= min(a, min(b, c)).\\

\textbf{Bài tập 1.8.5:} Prove using the notion of without loss of generality
that min(x, y) = (x + y − |x − y|)/2 and max(x, y) = (x + y + |x − y|)/2 whenever x and y are real numbers.\\
Because |x − y| = |y − x|, the values of x and y are interchangeable. Therefore, without loss of generality, we can assume that x ≥ y. Then (x + y − (x − y))/2 = (x + y − x + y)/2 = 2y/2 = y = min(x, y). Similarly, (x + y + (x − y))/2 = (x + y + x − y)/2 = 2x/2 = x = max(x, y).\\

\textbf{Bài tập 1.8.6:} Prove using the notion of without loss of generality that 5x + 5y is an odd integer when x and y are integers of opposite parity
The values of x and y are interchangeable. Therefore, without loss of generality, we can assume that x is even and y is odd. By the definition x = 2k + 1, y = 2n with n, k $\in Z^{+}$. Therefore, we have 5x +5y = 5(x + y) = 5(2k + 1 + 2n) = 10(m+n) + 4 + 1.\\
Because 10(m+n) + 4 + 1 is odd, we can conclude that 5x + 5y is an odd integer.\\

\textbf{ Bài tập 1.8.7:} 
Prove the triangle inequality, which states that if x and y are real numbers, then |x| + |y| $\ge$ |x + y| (where |x| represents the absolute value of x, which equals x if x $ge$ 0 and equals -x if x < 0).\\
We have three cases:\\
Cases(1), we assume that $x \ge 0, y \ge 0$. So that |x + y| = x + y, |x| + |y| = x + y. Therefore, $|x + y| \le |x| + |y|$\\
Cases(2), we assume that $x \ge 0, y < 0$. So that |x + y| = x + y, |x| + |y| = x - y. Therefore, $|x + y| \le |x| + |y|$.\\
Cases(3), we assume that x < 0, y < 0. So that |x + y| = -(x + y), |x| + |y| = -x - y = -(x + y). Therefore, $|x + y| \le |x| + |y|$.\\
The values of x and y are interchangeable. Therefore, without loss of generality, we do not need to check the case x < 0, $y \ge 0$. 

\textbf{Bài tập 1.8.8:} Prove that there is a positive integer that equals the sum of the positive integers not exceeding it. Is your proof constructive or nonconstructive?\\
We use a constructive existence proof.\\
We assume that all positive integer not exceeding x with x is a positive integer are in set A. We need to find a number satisfying the sum of  set A is equal to x. We find that x = 1 is that number. We have A={1} then the sum of set A = 1. Therefore we are done.

\textbf{Bài tập 1.8.9:} Prove that there are 100 consecutive positive integers that are not perfect squares. Is your proof constructive or nonconstructive?\\
We use a constructive existence proof.\\
We find that $50^{2} = 2500$ and $51^{2} = 2601$ so from 2501 to 2600, there is 100 consecutive positive integers that are not squares.\\

\textbf{Bài tập 1.8.10:} Prove that either $2 \times 10^{500} + 15$ or $2 \times 10^{500} + 16$ is not a perfect square. Is your proof constructive or nonconstructive?\\
We use a constructive existence proof.\\ 
We have that the difference of $n^{2}$ and $(n+1)^{2}$ is an odd integer that is 2n + 1. So that the distances increases when n increases. In the other hand, the only consecutive perfect squares integers are 0 and 1. Because  $(2 \times 10^{500} + 15) - (2 \times 10^{500} + 16) = 1$, they can not both be the perfect squares.\\

\textbf{Bài tập 1.8.11:}  Prove that there exists a pair of consecutive integers such that one of these integers is a perfect square and the other is a perfect cube.\\
We find that a pair of consecutive integers that is 8 and 9. Because $8 = 2^{3}$ and $9 = 3^{2}$.\\

\textbf{Bài tập 1.8.13:} Prove or disprove that there is a rational number x and an irrational number y such that xy is irrational.\\
Let x = 2, so x is rational and y = $\sqrt{2}$, so y is irrational. We have $x^{y} = 2^{\sqrt{2}}$.\\
Assume that $2^{\sqrt{2}}$ is irraltional, then we assume that x' = $2^{\sqrt{2}}$ and y' = $\frac{\sqrt{2}}{4}$. Then we have $x'^{y'} = (2^{\sqrt{2}})^{\frac{\sqrt{2}}{4}} = 2^{\frac{\sqrt{2} \times \sqrt{2}}{4}} = \sqrt{2}$ is irrational.We are done.\\
Assume that $2^{\sqrt{2}}$ is irrational. We are done.\\

\textbf{Bài tập 1.8.14:}  Prove or disprove that if a and b are rational numbers,
then $a^{b}$ is also rational.\\
Let a = 2. so a is rational and b = $\frac{1}{2}$, so b is rational. Then we have $a^{b} = 2^{\frac{1}{2}} = \sqrt{2}$ which is a irrational number. So we are done.\\

\textbf{Bài tập 1.8.17:} Suppose that a and b are odd integers with a = b. Show there is a unique integer c such that |a - c| = |b - c|.\\
The equation |a - c| = |b - c| is equivalent to the disjunction of two equations: a - c = b - c or a - c = -b + c. The first of these is equivalent to a = b, which contradicts the assumptions made in this problem, so the original equation is equivalent to a − c = −b + c. By adding b + c to both sides and dividing by 2, we see that this equation is equivalent to c = (a + b)/2. Thus, there is a unique solution.\\

\textbf{Bài tập 1.8.18:} Show that if r is an irrational number, there is a unique integer n such that the distance between r and n is less than 1/2.\\
As r is irrational number, we have r lies between two integers. So there exists an integer n such that n < r < n + 1.\\
Because r is irrational $r \ne n + \frac{1}{2}$, we have two cases:\\
Cases(1), r < $n + \frac{1}{2}$. So that 0< r - n < $\frac{1}{2}$, so the distance between n and r is less than $\frac{1}{2}$\\
Cases(2), r > $n + \frac{1}{2}$. So that, (n + 1) - r < $\frac{1}{2}$, so the distance between n + 1 and r is less than $\frac{1}{2}$\\

\textbf{Bài tập 1.8.19:} Show that if n is an odd integer, then there is a unique integer k such that n is the sum of k - 2 and k + 3.\\
Because n is an odd integer, n = 2m + 1 with m $\in$ Z. The sum of  k - 2 and k + 3 is 2k + 1. We have n is an odd integer equal to the sum of k - 2 and k + 3 if only k = m. We can see that there is a unique solution.\\

\textbf{Bài tập 1.8.20:} Prove that given a real number x there exist unique numbers n and $\epsilon$ such that x = n + $\epsilon$ , n is an integer, and 0 $\le$ $\epsilon$ < 1.\\
As x is a real number, there exists an unique integer n such that n $\le$ x < n + 1. Subtracting both sides from n, we have 0 $\le$ x - n < 1.\\
Assume $\epsilon$ = x - n. So that, 0 $\le \epsilon$ < 1 and x = n + $\epsilon$. We are done.\\

\textbf{Bài tập 1.8.21:} Prove that given a real number x there exist unique numbers n and  such that x = n - $\epsilon$, n is an integer, and 0 $\le$ $\epsilon$ < 1.\\
As x is a real number, there exists an unique integer n such that n is the smallest integer satisfying $x \le n$. So that differ of n and x is smaller than 1. Therefore, we have an inequation $0 \le n-x < 1$. Assume that $\epsilon$ is equal to n - x. So that $0 \le \epsilon < 1$ and x = n - $\epsilon$. We are done.\\

\textbf{Bài tập 1.8.22:} Use forward reasoning to show that if x is a nonzero real number, then $x^{2} + 1/x^{2} \ge 2$. [Hint: Start with the inequality $(x - 1/x)^{2} \ge 0$ which holds for all nonzero real numbers x.]\\
As x is a nonzero real number, there exists an real number 1/x. Then $(x - 1/x)^{2} \ge 0$. Because  $(x - 1/x)^{2} = x^{2} + \frac{1}{x^{2}} - 2x^{2} \times \frac{1}{x^{2}} =  x^{2} + \frac{1}{x^{2}} - 2$. This implies that $x^{2} + \frac{1}{x^{2}} -2 \ge 0$. Adding 2 to both sides, we obtain $x^{2} + \frac{1}{x^{2}} \ge 2$. We conclude that if x is a nonzero real number, then $x^{2} + \frac{1}{x^{2}} \ge 2$.\\

\textbf{Bài tập 1.8.24:} The quadratic mean of two real numbers x and y equals $\sqrt{(x^{2} + y^{2})/2}$. By computing the arithmetic and quadratic means of different pairs of positive real numbers, formulate a conjecture about their relative sizes and prove your conjecture.\\
The arithmetic mean of distinct positive real number x and y is always less than their quadratic mean. To prove $(x + y)/2 < \sqrt{(x^{2} + y^{2})/2}$, we square both sides of inequality and then we construct a sequence of equivalent inequalities to obtain $(x-y)^{2}/4 > 0$. The equipvalent inequalities are:\\
$\frac{(x + y)}{2} < \sqrt{\frac{(x^{2} + y^{2})}{2}}$\\
$\equiv \frac{(x + y)^{2}}{4} < \frac{x^{2} + y^{2}}{2}$\\
$\equiv \frac{x^{2} + 2xy + y^{2}}{4} < \frac{x^{2} + y^{2}}{2}$\\
$\equiv \frac{x^{2} + y^{2} - 2xy}{4} > 0$\\
$\equiv \frac{(x - y)^{2}}{4} > 0$\\
Because $(x - y)^{2} > 0$ when $x \ne y$, it follow that the final inequality is true. Because all of these inequalities are equivalent. It follows that $(x + y)/2 < \sqrt{(x^{2} + y^{2})/2}$ when $x \ne y$.\\
\clearpage

\section{Homework03a\_Sets\_Function.pdf}
\subsection{Bài tập bắt buộc}
\subsubsection{Bài tập 1}
\textbf{Đề bài:} 
\\\ \\\
\textbf{Lời giải:} \\\ \\\
\clearpage
\subsubsection{Bài tập 2}
\textbf{Đề bài:} 
\\\ \\\
\textbf{Lời giải:} \\\ \\\
\clearpage
\subsubsection{Bài tập 3}
\textbf{Đề bài:} 
\\\ \\\
\textbf{Lời giải:} \\\ \\\
\clearpage
\subsubsection{Bài tập 4}
\textbf{Đề bài:} 
\\\ \\\
\textbf{Lời giải:} \\\ \\\
\clearpage
\subsubsection{Bài tập 5} 
\textbf{Đề bài:} Let A and B be sets. Prove that P(A) $\cap$ P(B) = P(A $\cap$ B).
\\\ \\\
\textbf{Lời giải:} \\\ \\\
Let X $\in$ P(A $\cap$ B) then each elements of X is an element of A and B. So that, X also in P(A) and P(B). Therefore, X $\in$ P(A) $\cap$ P(B).\\
Let Y $\in$ P(A) and Y $\in$ P(B). So that, each elements of Y in an element of A and B. Hence, each elements of  is in A $\cap$ B. Therefore, Y $\in$ P(A $\cap$ B).
\clearpage
\subsubsection{Bài tập 6} 
\textbf{Đề bài:} Let A, B, and C be subsets of a universe U. Use definitions of set operations and set identities to prove the following equality of sets:\\
$((B \cap A) \cup (B \cap C)) \ (A \cap B \cap C) = (B \cap (A \triangle C))$
\\\ \\\
\textbf{Lời giải:} \\\ \\\
$((B \cap A) \cup (B \cap C)) \ (A \cap B \cap C)$\\
$\equiv (B \cap A) \ C \cup (B \cap C) \ (B \cap A)$\\
$\equiv (B \cap A) \ C \cup (B \cap C) \ A$\\
\\
$B \cap (A \triangle C)$\\
$\equiv B \cap [(A \ C) \cup (C \ A)]$\\
$\equiv (B \cap A) \ C \cup (B \cap C) \ A$\\
\clearpage
\subsubsection{Bài tập 7}
\textbf{Đề bài:} Let $A = \{a, b, c\}$, $B = \{1, 2, 3, 4\}$, and $C = \{\pi, \phi, i\}$. Define functions $f : A \rightarrow B$ and $g : B \rightarrow C$ as \\\
\begin{center}
$f(x) = \begin{cases} 2, & x=a \\ 3, & x = b \\ 4, & x=c \end{cases}$ \hspace{0.5cm}
$g(x) = \begin{cases} \pi, & x=1 \\ \phi, & x = 2 \\ i, & x=3 \\ \pi, & x=4 \end{cases}$
\end{center}
Consider each of the functions $f$, $g$, $g\circ f$ and determine if they are injective, surjective, or both. \\\ \\\
\textbf{Lời giải:} \\\ \\\
Xét ánh xạ $f : A \rightarrow B$. Ta thấy mọi ảnh của $f$ đều riêng biệt nên $f$ là đơn ánh. $f$ không phải toàn ánh vì với $x = 1$ thì không tồn tại $y \in A$ sao cho $f(y) = 1$. Vậy $f$ là đơn ánh. \\\ \\\
Xét ánh xạ $g : B \rightarrow C$. Ta thấy mọi phần tử $x \in C$ đều có nghịch ảnh trên $B$, nên $g$ là toàn ánh. $g$ không phải là đơn ánh vì $g(1) = g(4) = \pi$. Vậy $g$ là toàn ánh. \\\ \\\
Xét ánh xạ $g\circ f : A \rightarrow C$.
Từ hai ánh xạ $f$ và $g$, ta viết lại ánh xạ $g\circ f : A \rightarrow C$ thành: 
\begin{center}
$g\circ f = g(f(x)) = \begin{cases} \pi, & f(x)=1 \\ \phi, & f(x) = 2 \\ i, & f(x)=3 \\ \pi, & f(x)=4 \end{cases} = \begin{cases} \phi, & x = a \\ i, & x=b \\ \pi, & x=c \end{cases}$
\end{center}
Ta thấy mọi ảnh của $g \circ f$ đều phân biệt và mọi ảnh đều có nghịch ảnh tương ứng. Vậy $g \circ f$ là một song ánh.
\clearpage
\subsubsection{Bài tập 8}
\textbf{Đề bài:} 
\\\ \\\
\textbf{Lời giải:} \\\ \\\
\clearpage
\subsection{Bonus}
\textbf{Bài tập 2.2.18:} Let A and B be sets. Show that:
\begin{enumerate}
	\item[a)] $(A \cup B) \subseteq (A \cup B \cup C)$\\
	Let x be arbitrary\\
	 $x \in (A \cup B) \rightarrow x \in (A \cup B) \lor (x \in C) \rightarrow	x \in (A \cup B \cup C)$\\
	Thus since x $\in$ (A $\cup$ B) $\rightarrow$ $x \in (A \cup B \cup C)$. It follows that $(A \cup B) \subseteq (A \cup B \cup C)$, by definition of subset.
	\item[b)] $ (A \cap B \cap C) \subseteq (A \cap B)$\\
	Let x be arbitrary\\
	$x \in (A \cap B \cap C) \rightarrow (x \in (A \cap B)) \land (x \in C) \rightarrow x \in (A \cap B)$
	Thus since $x \in (A \cap B \cap C) \rightarrow x \in (A \cap B)$. It follows that $(A \cap B \cap C) \subseteq (A \cap B)$, by definition of subset.
	\item[c)] $(A - B) - C \subseteq A - C$
	Let x be arbitrary
	$x \in (A - B) - C \rightarrow x \in (A - B) \land (x \notin C)\\
	\rightarrow (x \in A) \land (x \notin B) \land (x \notin C)\\
	\rightarrow (x \in A) \land (x \notin C)
	\rightarrow x \in (A - C)$
	Thus since $x \in (A - B) - C \rightarrow x \in (A - C)$. It follows that $(A - B) - C \subseteq A - C$, by definition of subset.
	\item[d)] $(A - C) \cap (C - B) = \varnothing$\\
	Let x be arbitrary\\
	$x \in (A - C) \cap (C - B) \rightarrow (x \in (A - C)) \land (x \in (C - B))\\
	\rightarrow (x \in A) \land (x \notin C) \land (x \in C) \land (x \notin B)\\
	\rightarrow (x \in A) (x \in \varnothing) (x \notin B)
	\rightarrow x \in \varnothing$
	Thus since $x \in (A - C) \cap (C - B) \rightarrow x \in \varnothing$. It follows that $(A - C) \cap (C - B) = \varnothing$\\
\end{enumerate} 
\textbf{Bài tập 2.2.19:} Show that if A and B are sets, then\\
\begin{enumerate}
	\item[a)] $A - B = A \cap \overline{B}$
	$A - B ={x | (x \in A) \land (x \notin B)}\\
	 A - B ={x | (x \in A) \land x \in \overline{B}}\\
	 A \cap \overline{B}={x | x \in A \land x \in \overline{B}}\\
	\rightarrow A - B = A \cap \overline{B}$ 
	\item[b)] $(A \cap B) \cup (A \cap \overline{B}) = A$
		$(A \cap B) \cup (A \cap \overline{B} = A \cap (B \cup \overline{B}) = A \cap U = A\\
		\rightarrow (A \cap B) \cup (A \cap \overline{B}) = A$
\end{enumerate}
\textbf{Bài tập 2.2.20:} Show that if A and B are sets with A $\subseteq$ B, then
\begin{enumerate}
	\item[a)] $A \cup B = B$\\
	Let x be arbitrary\\
	Because A $\subseteq$ B, $x \in A \rightarrow x \in B$
	$A \cup B = {x | (x \in A) \lor (x \in B)} = {x | x \in B \lor x \in B} = {x | x \in B} = B$\\
	\item[b)] $A \cap B = A$\\
	Let x be arbitrary\\
	$A \cap B = {x | (x \in A) \land (x \in B)} = {x | x \in A} = A$\\
\end{enumerate}
\textbf{Bài tập 2.2.32:} Find the symmetric difference of {1, 3, 5} and {1, 2, 3}.\\
{1.3.5} $\oplus$ {1,2,3} = {5,2}\\
\textbf{Bài tập 2.2.33:} Find the symmetric difference of the set of computer science majors at a school and the set of mathematics majors at this school.\\
The set of students who are computer science major but not mathematics major or who are mathematics major but not computer science major.\\
\textbf{Bài tập 2.2.35} Show that $A \oplus B = (A \cup B) - (A \cap B)$\\
$(A \cup B) - (A \cap B)$: An element is in $(A \cup B) - (A \cap B)$ if it is in union of A and B but not in intersection of A and B which means that it is in either A or B but not both A and B. This is exacly the same as A $\oplus$ B.\\
\textbf{Bài tập 2.2.36:} Show that $A \oplus B = (A - B) \cup (B - A)$\\
An element is in $(A - B) \cup (B - A)$ if it is in a but not in B or it in B but not in A. Which means that it is in either A or B but not both A and B. This is exactly the same as A $\oplus$ B.\\
\textbf{Bài tập 2.2.37:} Show that if A is a subset of a universal set U, then
\begin{enumerate}
	\item[a)] $A \oplus A = \varnothing$
	$A \oplus A = (A \cup A) - (A \cap A) = A - A = \varnothing$\\
	\item[b)] $A \oplus \varnothing = A$
	$A \oplus \varnothing = (A \cup \varnothing) - (A \cap \varnothing) = A - \varnothing = A$\\
	\item[c)] $A \oplus U = \overline{A}$
	$A \oplus U = (A \cup U) - (A \cap U) = U - A = \overline{A}$
	\item[d)] $A \oplus \overline{A} = U$
	$A \oplus \overline{A} = (A - \overline{A}) \cup (\overline{A} - A) = A \cup \overline{A} = U$\\
\end{enumerate}
\textbf{Bài tập 2.2.38:} Show that if A and B are sets, then:
\begin{enumerate}
	\item[a)] $A \oplus B = B \oplus A$\\
	$A \oplus B = (A \cup B) - (A \cap B)$\\
	$B \oplus A = (B \cup A) - (B \cap A) = (A \cup B) - (A \cap B)$\\
	\item[b)] $(A \oplus B) \oplus B = A$\\
	$(A \oplus B) \oplus B = ((A \oplus B) \cup B) - ((A \oplus B) \cap B)\\
	= ((A-B) \cup (B-A) \cup B) - (B \cap ((A \cup B)-(B \cap A))\\ 
	= ((A \cup B) \cup (B - A)) - (((A - B) \cap B) \cup ((B - A) \cap B))\\
	= (A \cup B) - (\varnothing \cup (B - A))\\ 
	= (A \cup B) - (B - A)\\
	= A$
\end{enumerate}
\textbf{Bài tập 2.2.39:}  What can you say about the sets A and B if $A \oplus B = A$\\
$\rightarrow set B = \varnothing$
\clearpage

\section{Homework03b\_Sequences.pdf} 

\subsection{Bài tập bắt buộc}
\subsubsection{Bài tập 1}
\textbf{Đề bài:} 
\\\ \\\
\textbf{Lời giải:} \\\ \\\
\clearpage
\subsubsection{Bài tập 2}
\textbf{Đề bài:} 
\\\ \\\
\textbf{Lời giải:} \\\ \\\
\clearpage
\subsubsection{Bài tập 3}
\textbf{Đề bài:} Mathematical induction can be used to prove things that can be proven directly. In the two problems below, you will essentially prove the induction step of an induction proof.
a. For n ≥ 0, let P(n) denote the sentence " The 2×n chessboard can be tiled using dominoes." Prove P(n) $\rightarrow$ P(n + 1).
b. For n ≥ 0, let Q(n) denote the sentence " The 3 × 2n chessboard can be tiled using dominoes." Prove Q(n) $\rightarrow$ Q(n + 1).
\\\ \\\
\textbf{Lời giải:} \\\ \\\
\begin{enumerate}
	\item[a)] P(0) is always true.\\
	P(1) is true because the size of chessboard will be 2 $\times$ 1 is fit with a domino.
	Assume that P(k) is true for all $k \ge 0$ that is 2 $\times$ k chessboard can be tiled using dominoes. We need to prove that it is also true with P(k + 1).\\
	With P(k + 1), we will have 2 $\times$ (k + 1) chessboard. This can be demonstrated by (2 $\times$ k) + (2 $\times$ 1). Because we can tile 2 $\times$ k and 2 $\times$ 1 chessboard by using dominoes, we can also tile 2 $\times$ (k + 1) chessboard.\\
	\item[b)] P(0) is always true.\\
	P(1) is true because the size of chessboard will be 3 $\times$ 2 is fit with three dominoes.\\
	Assume that P(k) is true for all $k \ge 0$ that is 3 $\times$ 2k chessboard can be tiled using dominoes. We need to prove that it is also true with P(k + 1).\\
	With P(k + 1), we will have 3 $\times$ 2(k + 1) chessboard. This can be demonstrated by (3 $\times$ 2k) + (3 $\times$ 2). Because we can tile (3 $\times$ 2k) and (3 $\times$ 2) chessboard by using domninoes, we can also tile 3 $\times$ 2(k + 1) chessboard.\\   
\end{enumerate}
\clearpage
\subsubsection{Bài tập 4}
\textbf{Đề bài: }Define a sequence $\{f_n\}_{n=0}^\infty$ as $f_0 = 1$ and for $n \geq 1$, $f_{n+1} = \frac{1}{1+f_n}$. Prove that for $n \geq 0$, $f_n = \frac{F_{n+1}}{F_{n+2}}$, where $\{F_n\}_{n=0}^\infty$ is the Fibonacci sequence. \\\ \\\
\textbf{Lời giải: } \\\ \\\
Ta đi chứng minh quy nạp rằng $f_n = \frac{F_{n+1}}{F_{n+2}}$. (1)\\\
Với $n = 0$, ta có: $f_0 = 1 = \frac{1}{1} = \frac{F_1}{F_2}$. \\\
Giả sử đẳng thức (1) đúng với mọi $n = k \in \textbf{N}, k \geq 0$.
Ta chứng minh rằng đẳng thức (1) cũng đúng với $n = k+1$.\\\
Thật vậy, ta có: \\\
$f_{n+1} = \frac{1}{1+f_n} = \frac{1}{1+\frac{F_{n+1}}{F{n+2}}} = \frac{F_{n+2}}{F_{n+1} + F{n+2}} = \frac{F_{n+2}}{F_{n+3}}$. \\\
Theo nguyên lý quy nạp, ta có điều phải chứng minh.
\clearpage
\subsubsection{Bài tập 5}
\textbf{Đề bài:} 
\\\ \\\
\textbf{Lời giải:} \\\ \\\
\clearpage

\subsection{Bonus}
\clearpage

\section{Homework03c\_Sequences\_and\_Sums.pdf}
\subsection{Bài tập bắt buộc}
\subsubsection{Bài tập 1}
\textbf{Đề bài:} 
\\\ \\\
\textbf{Lời giải:} \\\ \\\
\clearpage
\subsubsection{Bài tập 2}
\textbf{Đề bài:} 
\\\ \\\
\textbf{Lời giải:} \\\ \\\
\clearpage
\subsubsection{Bài tập 3}
\textbf{Đề bài:} 
\\\ \\\
\textbf{Lời giải:} \\\ \\\
\clearpage
\subsubsection{Bài tập 4}
\textbf{Đề bài:} 
\\\ \\\
\textbf{Lời giải:} \\\ \\\
\clearpage
\subsubsection{Bài tập 5}
\textbf{Đề bài:} 
\\\ \\\
\textbf{Lời giải:} \\\ \\\
\clearpage

\clearpage



\end{document}

